%% Generated by Sphinx.
\def\sphinxdocclass{report}
\documentclass[letterpaper,10pt,english]{sphinxmanual}
\ifdefined\pdfpxdimen
   \let\sphinxpxdimen\pdfpxdimen\else\newdimen\sphinxpxdimen
\fi \sphinxpxdimen=.75bp\relax

\PassOptionsToPackage{warn}{textcomp}
\usepackage[utf8]{inputenc}
\ifdefined\DeclareUnicodeCharacter
% support both utf8 and utf8x syntaxes
  \ifdefined\DeclareUnicodeCharacterAsOptional
    \def\sphinxDUC#1{\DeclareUnicodeCharacter{"#1}}
  \else
    \let\sphinxDUC\DeclareUnicodeCharacter
  \fi
  \sphinxDUC{00A0}{\nobreakspace}
  \sphinxDUC{2500}{\sphinxunichar{2500}}
  \sphinxDUC{2502}{\sphinxunichar{2502}}
  \sphinxDUC{2514}{\sphinxunichar{2514}}
  \sphinxDUC{251C}{\sphinxunichar{251C}}
  \sphinxDUC{2572}{\textbackslash}
\fi
\usepackage{cmap}
\usepackage[T1]{fontenc}
\usepackage{amsmath,amssymb,amstext}
\usepackage{babel}



\usepackage{times}
\expandafter\ifx\csname T@LGR\endcsname\relax
\else
% LGR was declared as font encoding
  \substitutefont{LGR}{\rmdefault}{cmr}
  \substitutefont{LGR}{\sfdefault}{cmss}
  \substitutefont{LGR}{\ttdefault}{cmtt}
\fi
\expandafter\ifx\csname T@X2\endcsname\relax
  \expandafter\ifx\csname T@T2A\endcsname\relax
  \else
  % T2A was declared as font encoding
    \substitutefont{T2A}{\rmdefault}{cmr}
    \substitutefont{T2A}{\sfdefault}{cmss}
    \substitutefont{T2A}{\ttdefault}{cmtt}
  \fi
\else
% X2 was declared as font encoding
  \substitutefont{X2}{\rmdefault}{cmr}
  \substitutefont{X2}{\sfdefault}{cmss}
  \substitutefont{X2}{\ttdefault}{cmtt}
\fi


\usepackage[Bjarne]{fncychap}
\usepackage{sphinx}

\fvset{fontsize=\small}
\usepackage{geometry}

% Include hyperref last.
\usepackage{hyperref}
% Fix anchor placement for figures with captions.
\usepackage{hypcap}% it must be loaded after hyperref.
% Set up styles of URL: it should be placed after hyperref.
\urlstyle{same}

\usepackage{sphinxmessages}
\setcounter{tocdepth}{2}



\title{fdasrsf Documentation}
\date{Aug 08, 2019}
\release{1.5.0}
\author{J. Derek Tucker}
\newcommand{\sphinxlogo}{\vbox{}}
\renewcommand{\releasename}{Release}
\makeindex
\begin{document}

\pagestyle{empty}
\sphinxmaketitle
\pagestyle{plain}
\sphinxtableofcontents
\pagestyle{normal}
\phantomsection\label{\detokenize{index::doc}}


A python package for functional data analysis using the square root
slope framework and curves using the square root velocity framework
which performs pair-wise and group-wise alignment as well as modeling
using functional component analysis and regression.


\chapter{Functional Alignment}
\label{\detokenize{time_warping:module-time_warping}}\label{\detokenize{time_warping:functional-alignment}}\label{\detokenize{time_warping::doc}}\index{time\_warping (module)@\spxentry{time\_warping}\spxextra{module}}
Group-wise function alignment using SRSF framework and Dynamic Programming

moduleauthor:: Derek Tucker \textless{}\sphinxhref{mailto:jdtuck@sandia.gov}{jdtuck@sandia.gov}\textgreater{}
\index{align\_fPCA() (in module time\_warping)@\spxentry{align\_fPCA()}\spxextra{in module time\_warping}}

\begin{fulllineitems}
\phantomsection\label{\detokenize{time_warping:time_warping.align_fPCA}}\pysiglinewithargsret{\sphinxcode{\sphinxupquote{time\_warping.}}\sphinxbfcode{\sphinxupquote{align\_fPCA}}}{\emph{f}, \emph{time}, \emph{num\_comp=3}, \emph{showplot=True}, \emph{smoothdata=False}}{}
aligns a collection of functions while extracting principal components.
The functions are aligned to the principal components
\begin{quote}\begin{description}
\item[{Parameters}] \leavevmode\begin{itemize}
\item {} 
\sphinxstyleliteralstrong{\sphinxupquote{f}} (\sphinxstyleliteralemphasis{\sphinxupquote{np.ndarray}}) \textendash{} numpy ndarray of shape (M,N) of N functions with M samples

\item {} 
\sphinxstyleliteralstrong{\sphinxupquote{time}} (\sphinxstyleliteralemphasis{\sphinxupquote{np.ndarray}}) \textendash{} vector of size M describing the sample points

\item {} 
\sphinxstyleliteralstrong{\sphinxupquote{num\_comp}} \textendash{} number of fPCA components

\item {} 
\sphinxstyleliteralstrong{\sphinxupquote{showplot}} \textendash{} Shows plots of results using matplotlib (default = T)

\item {} 
\sphinxstyleliteralstrong{\sphinxupquote{smooth\_data}} (\sphinxhref{https://docs.python.org/3/library/functions.html\#bool}{\sphinxstyleliteralemphasis{\sphinxupquote{bool}}}) \textendash{} Smooth the data using a box filter (default = F)

\item {} 
\sphinxstyleliteralstrong{\sphinxupquote{sparam}} (\sphinxstyleliteralemphasis{\sphinxupquote{double}}) \textendash{} Number of times to run box filter (default = 25)

\end{itemize}

\item[{Return type}] \leavevmode
tuple of numpy array

\item[{Return fn}] \leavevmode
aligned functions - numpy ndarray of shape (M,N) of N
functions with M samples

\item[{Return qn}] \leavevmode
aligned srvfs - similar structure to fn

\item[{Return q0}] \leavevmode
original srvf - similar structure to fn

\item[{Return mqn}] \leavevmode
srvf mean or median - vector of length M

\item[{Return gam}] \leavevmode
warping functions - similar structure to fn

\item[{Return q\_pca}] \leavevmode
srsf principal directions

\item[{Return f\_pca}] \leavevmode
functional principal directions

\item[{Return latent}] \leavevmode
latent values

\item[{Return coef}] \leavevmode
coefficients

\item[{Return U}] \leavevmode
eigenvectors

\item[{Return orig\_var}] \leavevmode
Original Variance of Functions

\item[{Return amp\_var}] \leavevmode
Amplitude Variance

\item[{Return phase\_var}] \leavevmode
Phase Variance

\end{description}\end{quote}

\end{fulllineitems}

\index{align\_fPLS() (in module time\_warping)@\spxentry{align\_fPLS()}\spxextra{in module time\_warping}}

\begin{fulllineitems}
\phantomsection\label{\detokenize{time_warping:time_warping.align_fPLS}}\pysiglinewithargsret{\sphinxcode{\sphinxupquote{time\_warping.}}\sphinxbfcode{\sphinxupquote{align\_fPLS}}}{\emph{f}, \emph{g}, \emph{time}, \emph{comps=3}, \emph{showplot=True}, \emph{smoothdata=False}, \emph{delta=0.01}, \emph{max\_itr=100}}{}
This function aligns a collection of functions while performing
principal least squares
\begin{quote}\begin{description}
\item[{Parameters}] \leavevmode\begin{itemize}
\item {} 
\sphinxstyleliteralstrong{\sphinxupquote{f}} (\sphinxstyleliteralemphasis{\sphinxupquote{np.ndarray}}) \textendash{} numpy ndarray of shape (M,N) of N functions with M samples

\item {} 
\sphinxstyleliteralstrong{\sphinxupquote{g}} (\sphinxstyleliteralemphasis{\sphinxupquote{np.ndarray}}) \textendash{} numpy ndarray of shape (M,N) of N functions with M samples

\item {} 
\sphinxstyleliteralstrong{\sphinxupquote{time}} (\sphinxstyleliteralemphasis{\sphinxupquote{np.ndarray}}) \textendash{} vector of size M describing the sample points

\item {} 
\sphinxstyleliteralstrong{\sphinxupquote{comps}} \textendash{} number of fPLS components

\item {} 
\sphinxstyleliteralstrong{\sphinxupquote{showplot}} \textendash{} Shows plots of results using matplotlib (default = T)

\item {} 
\sphinxstyleliteralstrong{\sphinxupquote{smooth\_data}} (\sphinxhref{https://docs.python.org/3/library/functions.html\#bool}{\sphinxstyleliteralemphasis{\sphinxupquote{bool}}}) \textendash{} Smooth the data using a box filter (default = F)

\item {} 
\sphinxstyleliteralstrong{\sphinxupquote{delta}} \textendash{} gradient step size

\item {} 
\sphinxstyleliteralstrong{\sphinxupquote{max\_itr}} \textendash{} maximum number of iterations

\end{itemize}

\item[{Return type}] \leavevmode
tuple of numpy array

\item[{Return fn}] \leavevmode
aligned functions - numpy ndarray of shape (M,N) of N

\end{description}\end{quote}

functions with M samples
:return gn: aligned functions - numpy ndarray of shape (M,N) of N
functions with M samples
:return qfn: aligned srvfs - similar structure to fn
:return qgn: aligned srvfs - similar structure to fn
:return qf0: original srvf - similar structure to fn
:return qg0: original srvf - similar structure to fn
:return gam: warping functions - similar structure to fn
:return wqf: srsf principal weight functions
:return wqg: srsf principal weight functions
:return wf: srsf principal weight functions
:return wg: srsf principal weight functions
:return cost: cost function value

\end{fulllineitems}

\index{srsf\_align() (in module time\_warping)@\spxentry{srsf\_align()}\spxextra{in module time\_warping}}

\begin{fulllineitems}
\phantomsection\label{\detokenize{time_warping:time_warping.srsf_align}}\pysiglinewithargsret{\sphinxcode{\sphinxupquote{time\_warping.}}\sphinxbfcode{\sphinxupquote{srsf\_align}}}{\emph{f}, \emph{time}, \emph{method='mean'}, \emph{omethod='DP'}, \emph{showplot=True}, \emph{smoothdata=False}, \emph{parallel=False}, \emph{lam=0.0}}{}
This function aligns a collection of functions using the elastic
square-root slope (srsf) framework.
\begin{quote}\begin{description}
\item[{Parameters}] \leavevmode\begin{itemize}
\item {} 
\sphinxstyleliteralstrong{\sphinxupquote{f}} \textendash{} numpy ndarray of shape (M,N) of N functions with M samples

\item {} 
\sphinxstyleliteralstrong{\sphinxupquote{time}} \textendash{} vector of size M describing the sample points

\item {} 
\sphinxstyleliteralstrong{\sphinxupquote{method}} \textendash{} (string) warp calculate Karcher Mean or Median

\end{itemize}

\end{description}\end{quote}

(options = “mean” or “median”) (default=”mean”)
:param omethod: optimization method (DP, DP2, RBFGS) (default = DP)
:param showplot: Shows plots of results using matplotlib (default = T)
:param smoothdata: Smooth the data using a box filter (default = F)
:param parallel: run in parallel (default = F)
:param lam: controls the elasticity (default = 0)
:type lam: double
:type smoothdata: bool
:type f: np.ndarray
:type time: np.ndarray
\begin{quote}\begin{description}
\item[{Return type}] \leavevmode
tuple of numpy array

\item[{Return fn}] \leavevmode
aligned functions - numpy ndarray of shape (M,N) of N

\end{description}\end{quote}

functions with M samples
:return qn: aligned srvfs - similar structure to fn
:return q0: original srvf - similar structure to fn
:return fmean: function mean or median - vector of length M
:return mqn: srvf mean or median - vector of length M
:return gam: warping functions - similar structure to fn
:return orig\_var: Original Variance of Functions
:return amp\_var: Amplitude Variance
:return phase\_var: Phase Variance

Examples
\textgreater{}\textgreater{}\textgreater{} import tables
\textgreater{}\textgreater{}\textgreater{} fun=tables.open\_file(“../Data/simu\_data.h5”)
\textgreater{}\textgreater{}\textgreater{} f = fun.root.f{[}:{]}
\textgreater{}\textgreater{}\textgreater{} f = f.transpose()
\textgreater{}\textgreater{}\textgreater{} time = fun.root.time{[}:{]}
\textgreater{}\textgreater{}\textgreater{} out = srsf\_align(f,time)

\end{fulllineitems}

\index{srsf\_align\_pair() (in module time\_warping)@\spxentry{srsf\_align\_pair()}\spxextra{in module time\_warping}}

\begin{fulllineitems}
\phantomsection\label{\detokenize{time_warping:time_warping.srsf_align_pair}}\pysiglinewithargsret{\sphinxcode{\sphinxupquote{time\_warping.}}\sphinxbfcode{\sphinxupquote{srsf\_align\_pair}}}{\emph{f}, \emph{g}, \emph{time}, \emph{method='mean'}, \emph{showplot=True}, \emph{smoothdata=False}, \emph{lam=0.0}}{}
This function aligns a collection of functions using the elastic square-
root slope (srsf) framework.
\begin{quote}\begin{description}
\item[{Parameters}] \leavevmode\begin{itemize}
\item {} 
\sphinxstyleliteralstrong{\sphinxupquote{f}} (\sphinxstyleliteralemphasis{\sphinxupquote{np.ndarray}}) \textendash{} numpy ndarray of shape (M,N) of N functions with M samples

\item {} 
\sphinxstyleliteralstrong{\sphinxupquote{g}} \textendash{} numpy ndarray of shape (M,N) of N functions with M samples

\item {} 
\sphinxstyleliteralstrong{\sphinxupquote{time}} (\sphinxstyleliteralemphasis{\sphinxupquote{np.ndarray}}) \textendash{} vector of size M describing the sample points

\item {} 
\sphinxstyleliteralstrong{\sphinxupquote{method}} \textendash{} (string) warp calculate Karcher Mean or Median (options =
“mean” or “median”) (default=”mean”)

\item {} 
\sphinxstyleliteralstrong{\sphinxupquote{showplot}} \textendash{} Shows plots of results using matplotlib (default = T)

\item {} 
\sphinxstyleliteralstrong{\sphinxupquote{smoothdata}} (\sphinxhref{https://docs.python.org/3/library/functions.html\#bool}{\sphinxstyleliteralemphasis{\sphinxupquote{bool}}}) \textendash{} Smooth the data using a box filter (default = F)

\item {} 
\sphinxstyleliteralstrong{\sphinxupquote{lam}} (\sphinxstyleliteralemphasis{\sphinxupquote{double}}) \textendash{} controls the elasticity (default = 0)

\end{itemize}

\item[{Return type}] \leavevmode
tuple of numpy array

\item[{Return fn}] \leavevmode
aligned functions - numpy ndarray of shape (M,N) of N
functions with M samples

\item[{Return gn}] \leavevmode
aligned functions - numpy ndarray of shape (M,N) of N
functions with M samples

\item[{Return qfn}] \leavevmode
aligned srvfs - similar structure to fn

\item[{Return qgn}] \leavevmode
aligned srvfs - similar structure to fn

\item[{Return qf0}] \leavevmode
original srvf - similar structure to fn

\item[{Return qg0}] \leavevmode
original srvf - similar structure to fn

\item[{Return fmean}] \leavevmode
f function mean or median - vector of length N

\item[{Return gmean}] \leavevmode
g function mean or median - vector of length N

\item[{Return mqfn}] \leavevmode
srvf mean or median - vector of length N

\item[{Return mqgn}] \leavevmode
srvf mean or median - vector of length N

\item[{Return gam}] \leavevmode
warping functions - similar structure to fn

\end{description}\end{quote}

\end{fulllineitems}



\chapter{Functional Principal Component Analysis}
\label{\detokenize{fPCA:module-fPCA}}\label{\detokenize{fPCA:functional-principal-component-analysis}}\label{\detokenize{fPCA::doc}}\index{fPCA (module)@\spxentry{fPCA}\spxextra{module}}
Vertical and Horizontal Functional Principal Component Analysis using SRSF

moduleauthor:: Derek Tucker \textless{}\sphinxhref{mailto:jdtuck@sandia.gov}{jdtuck@sandia.gov}\textgreater{}
\index{horizfPCA() (in module fPCA)@\spxentry{horizfPCA()}\spxextra{in module fPCA}}

\begin{fulllineitems}
\phantomsection\label{\detokenize{fPCA:fPCA.horizfPCA}}\pysiglinewithargsret{\sphinxcode{\sphinxupquote{fPCA.}}\sphinxbfcode{\sphinxupquote{horizfPCA}}}{\emph{gam}, \emph{time}, \emph{no=2}, \emph{showplot=True}}{}
This function calculates horizontal functional principal component analysis on aligned data
\begin{quote}\begin{description}
\item[{Parameters}] \leavevmode\begin{itemize}
\item {} 
\sphinxstyleliteralstrong{\sphinxupquote{gam}} \textendash{} numpy ndarray of shape (M,N) of N warping functions

\item {} 
\sphinxstyleliteralstrong{\sphinxupquote{time}} \textendash{} vector of size M describing the sample points

\item {} 
\sphinxstyleliteralstrong{\sphinxupquote{no}} (\sphinxhref{https://docs.python.org/3/library/functions.html\#int}{\sphinxstyleliteralemphasis{\sphinxupquote{int}}}) \textendash{} number of components to extract (default = 2)

\item {} 
\sphinxstyleliteralstrong{\sphinxupquote{showplot}} (\sphinxhref{https://docs.python.org/3/library/functions.html\#bool}{\sphinxstyleliteralemphasis{\sphinxupquote{bool}}}) \textendash{} Shows plots of results using matplotlib (default = T)

\end{itemize}

\item[{Return type}] \leavevmode
tuple of numpy ndarray

\item[{Return q\_pca}] \leavevmode
srsf principal directions

\item[{Return f\_pca}] \leavevmode
functional principal directions

\item[{Return latent}] \leavevmode
latent values

\item[{Return coef}] \leavevmode
coefficients

\item[{Return U}] \leavevmode
eigenvectors

\end{description}\end{quote}

\end{fulllineitems}

\index{jointfPCA() (in module fPCA)@\spxentry{jointfPCA()}\spxextra{in module fPCA}}

\begin{fulllineitems}
\phantomsection\label{\detokenize{fPCA:fPCA.jointfPCA}}\pysiglinewithargsret{\sphinxcode{\sphinxupquote{fPCA.}}\sphinxbfcode{\sphinxupquote{jointfPCA}}}{\emph{fn}, \emph{time}, \emph{qn}, \emph{q0}, \emph{gam}, \emph{no=2}, \emph{showplot=True}}{}
This function calculates joint functional principal component analysis
on aligned data
\begin{quote}\begin{description}
\item[{Parameters}] \leavevmode\begin{itemize}
\item {} 
\sphinxstyleliteralstrong{\sphinxupquote{fn}} \textendash{} numpy ndarray of shape (M,N) of N aligned functions with M
samples

\item {} 
\sphinxstyleliteralstrong{\sphinxupquote{time}} \textendash{} vector of size N describing the sample points

\item {} 
\sphinxstyleliteralstrong{\sphinxupquote{qn}} \textendash{} numpy ndarray of shape (M,N) of N aligned SRSF with M samples

\item {} 
\sphinxstyleliteralstrong{\sphinxupquote{no}} (\sphinxhref{https://docs.python.org/3/library/functions.html\#int}{\sphinxstyleliteralemphasis{\sphinxupquote{int}}}) \textendash{} number of components to extract (default = 2)

\item {} 
\sphinxstyleliteralstrong{\sphinxupquote{showplot}} (\sphinxhref{https://docs.python.org/3/library/functions.html\#bool}{\sphinxstyleliteralemphasis{\sphinxupquote{bool}}}) \textendash{} Shows plots of results using matplotlib (default = T)

\end{itemize}

\item[{Return type}] \leavevmode
tuple of numpy ndarray

\item[{Return q\_pca}] \leavevmode
srsf principal directions

\item[{Return f\_pca}] \leavevmode
functional principal directions

\item[{Return latent}] \leavevmode
latent values

\item[{Return coef}] \leavevmode
coefficients

\item[{Return U}] \leavevmode
eigenvectors

\end{description}\end{quote}

\end{fulllineitems}

\index{vertfPCA() (in module fPCA)@\spxentry{vertfPCA()}\spxextra{in module fPCA}}

\begin{fulllineitems}
\phantomsection\label{\detokenize{fPCA:fPCA.vertfPCA}}\pysiglinewithargsret{\sphinxcode{\sphinxupquote{fPCA.}}\sphinxbfcode{\sphinxupquote{vertfPCA}}}{\emph{fn}, \emph{time}, \emph{qn}, \emph{no=2}, \emph{showplot=True}}{}
This function calculates vertical functional principal component analysis
on aligned data
\begin{quote}\begin{description}
\item[{Parameters}] \leavevmode\begin{itemize}
\item {} 
\sphinxstyleliteralstrong{\sphinxupquote{fn}} \textendash{} numpy ndarray of shape (M,N) of N aligned functions with M
samples

\item {} 
\sphinxstyleliteralstrong{\sphinxupquote{time}} \textendash{} vector of size N describing the sample points

\item {} 
\sphinxstyleliteralstrong{\sphinxupquote{qn}} \textendash{} numpy ndarray of shape (M,N) of N aligned SRSF with M samples

\item {} 
\sphinxstyleliteralstrong{\sphinxupquote{no}} (\sphinxhref{https://docs.python.org/3/library/functions.html\#int}{\sphinxstyleliteralemphasis{\sphinxupquote{int}}}) \textendash{} number of components to extract (default = 2)

\item {} 
\sphinxstyleliteralstrong{\sphinxupquote{showplot}} (\sphinxhref{https://docs.python.org/3/library/functions.html\#bool}{\sphinxstyleliteralemphasis{\sphinxupquote{bool}}}) \textendash{} Shows plots of results using matplotlib (default = T)

\end{itemize}

\item[{Return type}] \leavevmode
tuple of numpy ndarray

\item[{Return q\_pca}] \leavevmode
srsf principal directions

\item[{Return f\_pca}] \leavevmode
functional principal directions

\item[{Return latent}] \leavevmode
latent values

\item[{Return coef}] \leavevmode
coefficients

\item[{Return U}] \leavevmode
eigenvectors

\end{description}\end{quote}

\end{fulllineitems}



\chapter{Elastic Functional Boxplots}
\label{\detokenize{boxplots:module-boxplots}}\label{\detokenize{boxplots:elastic-functional-boxplots}}\label{\detokenize{boxplots::doc}}\index{boxplots (module)@\spxentry{boxplots}\spxextra{module}}
Elastic Functional Boxplots

moduleauthor:: Derek Tucker \textless{}\sphinxhref{mailto:jdtuck@sandia.gov}{jdtuck@sandia.gov}\textgreater{}
\index{ampbox() (in module boxplots)@\spxentry{ampbox()}\spxextra{in module boxplots}}

\begin{fulllineitems}
\phantomsection\label{\detokenize{boxplots:boxplots.ampbox}}\pysiglinewithargsret{\sphinxcode{\sphinxupquote{boxplots.}}\sphinxbfcode{\sphinxupquote{ampbox}}}{\emph{ft}, \emph{f\_median}, \emph{qt}, \emph{q\_median}, \emph{time}, \emph{alpha=0.05}, \emph{k\_a=1}}{}
This function constructs the amplitude boxplot using the elastic
square-root slope (srsf) framework.
\begin{quote}\begin{description}
\item[{Parameters}] \leavevmode\begin{itemize}
\item {} 
\sphinxstyleliteralstrong{\sphinxupquote{ft}} \textendash{} numpy ndarray of shape (M,N) of N functions with M samples

\item {} 
\sphinxstyleliteralstrong{\sphinxupquote{f\_median}} \textendash{} vector of size M describing the median

\item {} 
\sphinxstyleliteralstrong{\sphinxupquote{qt}} \textendash{} numpy ndarray of shape (M,N) of N srsf functions with M samples

\item {} 
\sphinxstyleliteralstrong{\sphinxupquote{q\_median}} \textendash{} vector of size M describing the srsf median

\item {} 
\sphinxstyleliteralstrong{\sphinxupquote{time}} \textendash{} vector of size M describing the time

\item {} 
\sphinxstyleliteralstrong{\sphinxupquote{alpha}} \textendash{} quantile value (e.g.,=.05, i.e., 95\%)

\item {} 
\sphinxstyleliteralstrong{\sphinxupquote{k\_a}} \textendash{} scalar for outlier cutoff (e.g.,=1)

\end{itemize}

\item[{Return type}] \leavevmode
tuple of numpy array

\item[{Return fn}] \leavevmode
aligned functions - numpy ndarray of shape (M,N) of N

\end{description}\end{quote}

functions with M samples
:return Q1: First quartile
:return Q3: Second quartile
:return Q1a: First quantile based on alpha
:return Q3a: Second quantile based on alpha
:return minn: minimum extreme function
:return maxx: maximum extreme function
:return outlier\_index: indexes of outlier functions
:return f\_median: median function
:return q\_median: median srsf
:return plt: surface plot mesh

\end{fulllineitems}

\index{phbox() (in module boxplots)@\spxentry{phbox()}\spxextra{in module boxplots}}

\begin{fulllineitems}
\phantomsection\label{\detokenize{boxplots:boxplots.phbox}}\pysiglinewithargsret{\sphinxcode{\sphinxupquote{boxplots.}}\sphinxbfcode{\sphinxupquote{phbox}}}{\emph{gam}, \emph{time}, \emph{alpha=0.05}, \emph{k\_a=1}}{}
This function constructs phase boxplot for functional data using the elastic
square-root slope (srsf) framework.
\begin{quote}\begin{description}
\item[{Parameters}] \leavevmode\begin{itemize}
\item {} 
\sphinxstyleliteralstrong{\sphinxupquote{gam}} \textendash{} numpy ndarray of shape (M,N) of N warping functions with M samples

\item {} 
\sphinxstyleliteralstrong{\sphinxupquote{alpha}} \textendash{} quantile value (e.g.,=.05, i.e., 95\%)

\item {} 
\sphinxstyleliteralstrong{\sphinxupquote{k\_a}} \textendash{} scalar for outlier cutoff (e.g.,=1)

\end{itemize}

\item[{Return type}] \leavevmode
tuple of numpy array

\item[{Return fn}] \leavevmode
aligned functions - numpy ndarray of shape (M,N) of N

\end{description}\end{quote}

functions with M samples
:return Q1: First quartile
:return Q3: Second quartile
:return Q1a: First quantile based on alpha
:return Q3a: Second quantile based on alpha
:return minn: minimum extreme function
:return maxx: maximum extreme function
:return outlier\_index: indexes of outlier functions
:return median\_x: median warping function
:return psi\_median: median srsf of warping function
:return plt: surface plot mesh

\end{fulllineitems}



\chapter{Gaussian Generative Models}
\label{\detokenize{gauss_model:module-gauss_model}}\label{\detokenize{gauss_model:gaussian-generative-models}}\label{\detokenize{gauss_model::doc}}\index{gauss\_model (module)@\spxentry{gauss\_model}\spxextra{module}}
Gaussian Model of functional data

moduleauthor:: Derek Tucker \textless{}\sphinxhref{mailto:jdtuck@sandia.gov}{jdtuck@sandia.gov}\textgreater{}
\index{gauss\_model() (in module gauss\_model)@\spxentry{gauss\_model()}\spxextra{in module gauss\_model}}

\begin{fulllineitems}
\phantomsection\label{\detokenize{gauss_model:gauss_model.gauss_model}}\pysiglinewithargsret{\sphinxcode{\sphinxupquote{gauss\_model.}}\sphinxbfcode{\sphinxupquote{gauss\_model}}}{\emph{fn}, \emph{time}, \emph{qn}, \emph{gam}, \emph{n=1}, \emph{sort\_samples=False}}{}
This function models the functional data using a Gaussian model
extracted from the principal components of the srvfs
\begin{quote}\begin{description}
\item[{Parameters}] \leavevmode\begin{itemize}
\item {} 
\sphinxstyleliteralstrong{\sphinxupquote{fn}} (\sphinxstyleliteralemphasis{\sphinxupquote{np.ndarray}}) \textendash{} numpy ndarray of shape (M,N) of N aligned functions with
M samples

\item {} 
\sphinxstyleliteralstrong{\sphinxupquote{time}} (\sphinxstyleliteralemphasis{\sphinxupquote{np.ndarray}}) \textendash{} vector of size M describing the sample points

\item {} 
\sphinxstyleliteralstrong{\sphinxupquote{qn}} (\sphinxstyleliteralemphasis{\sphinxupquote{np.ndarray}}) \textendash{} numpy ndarray of shape (M,N) of N aligned srvfs with M samples

\item {} 
\sphinxstyleliteralstrong{\sphinxupquote{gam}} (\sphinxstyleliteralemphasis{\sphinxupquote{np.ndarray}}) \textendash{} warping functions

\item {} 
\sphinxstyleliteralstrong{\sphinxupquote{n}} (\sphinxstyleliteralemphasis{\sphinxupquote{integer}}) \textendash{} number of random samples

\item {} 
\sphinxstyleliteralstrong{\sphinxupquote{sort\_samples}} (\sphinxhref{https://docs.python.org/3/library/functions.html\#bool}{\sphinxstyleliteralemphasis{\sphinxupquote{bool}}}) \textendash{} sort samples (default = T)

\end{itemize}

\item[{Return type}] \leavevmode
tuple of numpy array

\item[{Return fs}] \leavevmode
random aligned samples

\item[{Return gams}] \leavevmode
random warping functions

\item[{Return ft}] \leavevmode
random samples

\end{description}\end{quote}

\end{fulllineitems}

\index{joint\_gauss\_model() (in module gauss\_model)@\spxentry{joint\_gauss\_model()}\spxextra{in module gauss\_model}}

\begin{fulllineitems}
\phantomsection\label{\detokenize{gauss_model:gauss_model.joint_gauss_model}}\pysiglinewithargsret{\sphinxcode{\sphinxupquote{gauss\_model.}}\sphinxbfcode{\sphinxupquote{joint\_gauss\_model}}}{\emph{fn}, \emph{time}, \emph{qn}, \emph{gam}, \emph{q0}, \emph{n=1}, \emph{no=3}}{}
This function models the functional data using a joint Gaussian model
extracted from the principal components of the srsfs
\begin{quote}\begin{description}
\item[{Parameters}] \leavevmode\begin{itemize}
\item {} 
\sphinxstyleliteralstrong{\sphinxupquote{fn}} (\sphinxstyleliteralemphasis{\sphinxupquote{np.ndarray}}) \textendash{} numpy ndarray of shape (M,N) of N aligned functions with
M samples

\item {} 
\sphinxstyleliteralstrong{\sphinxupquote{time}} (\sphinxstyleliteralemphasis{\sphinxupquote{np.ndarray}}) \textendash{} vector of size M describing the sample points

\item {} 
\sphinxstyleliteralstrong{\sphinxupquote{qn}} (\sphinxstyleliteralemphasis{\sphinxupquote{np.ndarray}}) \textendash{} numpy ndarray of shape (M,N) of N aligned srsfs with M samples

\item {} 
\sphinxstyleliteralstrong{\sphinxupquote{gam}} (\sphinxstyleliteralemphasis{\sphinxupquote{np.ndarray}}) \textendash{} warping functions

\item {} 
\sphinxstyleliteralstrong{\sphinxupquote{q0}} \textendash{} numpy ndarray of shape (M,N) of N unaligned srsfs with  samples

\item {} 
\sphinxstyleliteralstrong{\sphinxupquote{n}} (\sphinxstyleliteralemphasis{\sphinxupquote{integer}}) \textendash{} number of random samples

\item {} 
\sphinxstyleliteralstrong{\sphinxupquote{n}} \textendash{} number of principal components (default = 3)

\end{itemize}

\item[{Return type}] \leavevmode
tuple of numpy array

\item[{Return fs}] \leavevmode
random aligned samples

\item[{Return gams}] \leavevmode
random warping functions

\item[{Return ft}] \leavevmode
random samples

\end{description}\end{quote}

\end{fulllineitems}



\chapter{Functional Principal Least Squares}
\label{\detokenize{fPLS:module-fPLS}}\label{\detokenize{fPLS:functional-principal-least-squares}}\label{\detokenize{fPLS::doc}}\index{fPLS (module)@\spxentry{fPLS}\spxextra{module}}
Partial Least Squares using SVD

moduleauthor:: Derek Tucker \textless{}\sphinxhref{mailto:jdtuck@sandia.gov}{jdtuck@sandia.gov}\textgreater{}
\index{pls\_svd() (in module fPLS)@\spxentry{pls\_svd()}\spxextra{in module fPLS}}

\begin{fulllineitems}
\phantomsection\label{\detokenize{fPLS:fPLS.pls_svd}}\pysiglinewithargsret{\sphinxcode{\sphinxupquote{fPLS.}}\sphinxbfcode{\sphinxupquote{pls\_svd}}}{\emph{time}, \emph{qf}, \emph{qg}, \emph{no}, \emph{alpha=0.0}}{}
This function computes the partial least squares using SVD
\begin{quote}\begin{description}
\item[{Parameters}] \leavevmode\begin{itemize}
\item {} 
\sphinxstyleliteralstrong{\sphinxupquote{time}} \textendash{} vector describing time samples

\item {} 
\sphinxstyleliteralstrong{\sphinxupquote{qf}} \textendash{} numpy ndarray of shape (M,N) of N functions with M samples

\item {} 
\sphinxstyleliteralstrong{\sphinxupquote{qg}} \textendash{} numpy ndarray of shape (M,N) of N functions with M samples

\item {} 
\sphinxstyleliteralstrong{\sphinxupquote{no}} \textendash{} number of components

\item {} 
\sphinxstyleliteralstrong{\sphinxupquote{alpha}} \textendash{} amount of smoothing (Default = 0.0 i.e., none)

\end{itemize}

\item[{Return type}] \leavevmode
numpy ndarray

\item[{Return wqf}] \leavevmode
f weight function

\item[{Return wqg}] \leavevmode
g weight function

\item[{Return alpha}] \leavevmode
smoothing value

\item[{Return values}] \leavevmode
singular values

\end{description}\end{quote}

\end{fulllineitems}



\chapter{Elastic Regression}
\label{\detokenize{regression:module-regression}}\label{\detokenize{regression:elastic-regression}}\label{\detokenize{regression::doc}}\index{regression (module)@\spxentry{regression}\spxextra{module}}
Warping Invariant Regression using SRSF

moduleauthor:: Derek Tucker \textless{}\sphinxhref{mailto:jdtuck@sandia.gov}{jdtuck@sandia.gov}\textgreater{}
\index{elastic\_logistic() (in module regression)@\spxentry{elastic\_logistic()}\spxextra{in module regression}}

\begin{fulllineitems}
\phantomsection\label{\detokenize{regression:regression.elastic_logistic}}\pysiglinewithargsret{\sphinxcode{\sphinxupquote{regression.}}\sphinxbfcode{\sphinxupquote{elastic\_logistic}}}{\emph{f}, \emph{y}, \emph{time}, \emph{B=None}, \emph{df=20}, \emph{max\_itr=20}, \emph{cores=-1}, \emph{smooth=False}}{}
This function identifies a logistic regression model with
phase-variablity using elastic methods
\begin{quote}\begin{description}
\item[{Parameters}] \leavevmode\begin{itemize}
\item {} 
\sphinxstyleliteralstrong{\sphinxupquote{f}} (\sphinxstyleliteralemphasis{\sphinxupquote{np.ndarray}}) \textendash{} numpy ndarray of shape (M,N) of N functions with M samples

\item {} 
\sphinxstyleliteralstrong{\sphinxupquote{y}} \textendash{} numpy array of labels (1/-1)

\item {} 
\sphinxstyleliteralstrong{\sphinxupquote{time}} (\sphinxstyleliteralemphasis{\sphinxupquote{np.ndarray}}) \textendash{} vector of size M describing the sample points

\item {} 
\sphinxstyleliteralstrong{\sphinxupquote{B}} \textendash{} optional matrix describing Basis elements

\item {} 
\sphinxstyleliteralstrong{\sphinxupquote{df}} \textendash{} number of degrees of freedom B-spline (default 20)

\item {} 
\sphinxstyleliteralstrong{\sphinxupquote{max\_itr}} \textendash{} maximum number of iterations (default 20)

\item {} 
\sphinxstyleliteralstrong{\sphinxupquote{cores}} \textendash{} number of cores for parallel processing (default all)

\end{itemize}

\item[{Return type}] \leavevmode
tuple of numpy array

\item[{Return alpha}] \leavevmode
alpha parameter of model

\item[{Return beta}] \leavevmode
beta(t) of model

\item[{Return fn}] \leavevmode
aligned functions - numpy ndarray of shape (M,N) of M

\end{description}\end{quote}

functions with N samples
:return qn: aligned srvfs - similar structure to fn
:return gamma: calculated warping functions
:return q: original training SRSFs
:return B: basis matrix
:return b: basis coefficients
:return Loss: logistic loss

\end{fulllineitems}

\index{elastic\_mlogistic() (in module regression)@\spxentry{elastic\_mlogistic()}\spxextra{in module regression}}

\begin{fulllineitems}
\phantomsection\label{\detokenize{regression:regression.elastic_mlogistic}}\pysiglinewithargsret{\sphinxcode{\sphinxupquote{regression.}}\sphinxbfcode{\sphinxupquote{elastic\_mlogistic}}}{\emph{f}, \emph{y}, \emph{time}, \emph{B=None}, \emph{df=20}, \emph{max\_itr=20}, \emph{cores=-1}, \emph{delta=0.01}, \emph{parallel=True}, \emph{smooth=False}}{}
This function identifies a multinomial logistic regression model with
phase-variablity using elastic methods
\begin{quote}\begin{description}
\item[{Parameters}] \leavevmode\begin{itemize}
\item {} 
\sphinxstyleliteralstrong{\sphinxupquote{f}} (\sphinxstyleliteralemphasis{\sphinxupquote{np.ndarray}}) \textendash{} numpy ndarray of shape (M,N) of N functions with M samples

\item {} 
\sphinxstyleliteralstrong{\sphinxupquote{y}} \textendash{} numpy array of labels \{1,2,…,m\} for m classes

\item {} 
\sphinxstyleliteralstrong{\sphinxupquote{time}} (\sphinxstyleliteralemphasis{\sphinxupquote{np.ndarray}}) \textendash{} vector of size M describing the sample points

\item {} 
\sphinxstyleliteralstrong{\sphinxupquote{B}} \textendash{} optional matrix describing Basis elements

\item {} 
\sphinxstyleliteralstrong{\sphinxupquote{df}} \textendash{} number of degrees of freedom B-spline (default 20)

\item {} 
\sphinxstyleliteralstrong{\sphinxupquote{max\_itr}} \textendash{} maximum number of iterations (default 20)

\item {} 
\sphinxstyleliteralstrong{\sphinxupquote{cores}} \textendash{} number of cores for parallel processing (default all)

\end{itemize}

\item[{Return type}] \leavevmode
tuple of numpy array

\item[{Return alpha}] \leavevmode
alpha parameter of model

\item[{Return beta}] \leavevmode
beta(t) of model

\item[{Return fn}] \leavevmode
aligned functions - numpy ndarray of shape (M,N) of N

\end{description}\end{quote}

functions with M samples
:return qn: aligned srvfs - similar structure to fn
:return gamma: calculated warping functions
:return q: original training SRSFs
:return B: basis matrix
:return b: basis coefficients
:return Loss: logistic loss

\end{fulllineitems}

\index{elastic\_prediction() (in module regression)@\spxentry{elastic\_prediction()}\spxextra{in module regression}}

\begin{fulllineitems}
\phantomsection\label{\detokenize{regression:regression.elastic_prediction}}\pysiglinewithargsret{\sphinxcode{\sphinxupquote{regression.}}\sphinxbfcode{\sphinxupquote{elastic\_prediction}}}{\emph{f}, \emph{time}, \emph{model}, \emph{y=None}, \emph{smooth=False}}{}
This function performs prediction from an elastic regression model
with phase-variablity
\begin{quote}\begin{description}
\item[{Parameters}] \leavevmode\begin{itemize}
\item {} 
\sphinxstyleliteralstrong{\sphinxupquote{f}} \textendash{} numpy ndarray of shape (M,N) of N functions with M samples

\item {} 
\sphinxstyleliteralstrong{\sphinxupquote{time}} \textendash{} vector of size M describing the sample points

\item {} 
\sphinxstyleliteralstrong{\sphinxupquote{model}} \textendash{} indentified model from elastic\_regression

\item {} 
\sphinxstyleliteralstrong{\sphinxupquote{y}} \textendash{} truth, optional used to calculate SSE

\end{itemize}

\item[{Return type}] \leavevmode
tuple of numpy array

\item[{Return alpha}] \leavevmode
alpha parameter of model

\item[{Return beta}] \leavevmode
beta(t) of model

\item[{Return fn}] \leavevmode
aligned functions - numpy ndarray of shape (M,N) of N

\end{description}\end{quote}

functions with M samples
:return qn: aligned srvfs - similar structure to fn
:return gamma: calculated warping functions
:return q: original training SRSFs
:return B: basis matrix
:return b: basis coefficients
:return SSE: sum of squared error

\end{fulllineitems}

\index{elastic\_regression() (in module regression)@\spxentry{elastic\_regression()}\spxextra{in module regression}}

\begin{fulllineitems}
\phantomsection\label{\detokenize{regression:regression.elastic_regression}}\pysiglinewithargsret{\sphinxcode{\sphinxupquote{regression.}}\sphinxbfcode{\sphinxupquote{elastic\_regression}}}{\emph{f}, \emph{y}, \emph{time}, \emph{B=None}, \emph{lam=0}, \emph{df=20}, \emph{max\_itr=20}, \emph{cores=-1}, \emph{smooth=False}}{}
This function identifies a regression model with phase-variablity
using elastic methods
\begin{quote}\begin{description}
\item[{Parameters}] \leavevmode\begin{itemize}
\item {} 
\sphinxstyleliteralstrong{\sphinxupquote{f}} (\sphinxstyleliteralemphasis{\sphinxupquote{np.ndarray}}) \textendash{} numpy ndarray of shape (M,N) of N functions with M samples

\item {} 
\sphinxstyleliteralstrong{\sphinxupquote{y}} \textendash{} numpy array of N responses

\item {} 
\sphinxstyleliteralstrong{\sphinxupquote{time}} (\sphinxstyleliteralemphasis{\sphinxupquote{np.ndarray}}) \textendash{} vector of size M describing the sample points

\item {} 
\sphinxstyleliteralstrong{\sphinxupquote{B}} \textendash{} optional matrix describing Basis elements

\item {} 
\sphinxstyleliteralstrong{\sphinxupquote{lam}} \textendash{} regularization parameter (default 0)

\item {} 
\sphinxstyleliteralstrong{\sphinxupquote{df}} \textendash{} number of degrees of freedom B-spline (default 20)

\item {} 
\sphinxstyleliteralstrong{\sphinxupquote{max\_itr}} \textendash{} maximum number of iterations (default 20)

\item {} 
\sphinxstyleliteralstrong{\sphinxupquote{cores}} \textendash{} number of cores for parallel processing (default all)

\end{itemize}

\item[{Return type}] \leavevmode
tuple of numpy array

\item[{Return alpha}] \leavevmode
alpha parameter of model

\item[{Return beta}] \leavevmode
beta(t) of model

\item[{Return fn}] \leavevmode
aligned functions - numpy ndarray of shape (M,N) of M

\end{description}\end{quote}

functions with N samples
:return qn: aligned srvfs - similar structure to fn
:return gamma: calculated warping functions
:return q: original training SRSFs
:return B: basis matrix
:return b: basis coefficients
:return SSE: sum of squared error

\end{fulllineitems}

\index{logistic\_warp() (in module regression)@\spxentry{logistic\_warp()}\spxextra{in module regression}}

\begin{fulllineitems}
\phantomsection\label{\detokenize{regression:regression.logistic_warp}}\pysiglinewithargsret{\sphinxcode{\sphinxupquote{regression.}}\sphinxbfcode{\sphinxupquote{logistic\_warp}}}{\emph{beta}, \emph{time}, \emph{q}, \emph{y}}{}
calculates optimal warping for function logistic regression
\begin{quote}\begin{description}
\item[{Parameters}] \leavevmode\begin{itemize}
\item {} 
\sphinxstyleliteralstrong{\sphinxupquote{beta}} \textendash{} numpy ndarray of shape (M,N) of N functions with M samples

\item {} 
\sphinxstyleliteralstrong{\sphinxupquote{time}} \textendash{} vector of size N describing the sample points

\item {} 
\sphinxstyleliteralstrong{\sphinxupquote{q}} \textendash{} numpy ndarray of shape (M,N) of N functions with M samples

\item {} 
\sphinxstyleliteralstrong{\sphinxupquote{y}} \textendash{} numpy ndarray of shape (1,N) responses

\end{itemize}

\item[{Return type}] \leavevmode
numpy array

\item[{Return gamma}] \leavevmode
warping function

\end{description}\end{quote}

\end{fulllineitems}

\index{logit\_gradient() (in module regression)@\spxentry{logit\_gradient()}\spxextra{in module regression}}

\begin{fulllineitems}
\phantomsection\label{\detokenize{regression:regression.logit_gradient}}\pysiglinewithargsret{\sphinxcode{\sphinxupquote{regression.}}\sphinxbfcode{\sphinxupquote{logit\_gradient}}}{\emph{b}, \emph{X}, \emph{y}}{}
calculates gradient of the logistic loss
\begin{quote}\begin{description}
\item[{Parameters}] \leavevmode\begin{itemize}
\item {} 
\sphinxstyleliteralstrong{\sphinxupquote{b}} \textendash{} numpy ndarray of shape (M,N) of N functions with M samples

\item {} 
\sphinxstyleliteralstrong{\sphinxupquote{X}} \textendash{} numpy ndarray of shape (M,N) of N functions with M samples

\item {} 
\sphinxstyleliteralstrong{\sphinxupquote{y}} \textendash{} numpy ndarray of shape (1,N) responses

\end{itemize}

\item[{Return type}] \leavevmode
numpy array

\item[{Return grad}] \leavevmode
gradient of logisitc loss

\end{description}\end{quote}

\end{fulllineitems}

\index{logit\_hessian() (in module regression)@\spxentry{logit\_hessian()}\spxextra{in module regression}}

\begin{fulllineitems}
\phantomsection\label{\detokenize{regression:regression.logit_hessian}}\pysiglinewithargsret{\sphinxcode{\sphinxupquote{regression.}}\sphinxbfcode{\sphinxupquote{logit\_hessian}}}{\emph{s}, \emph{b}, \emph{X}, \emph{y}}{}
calculates hessian of the logistic loss
\begin{quote}\begin{description}
\item[{Parameters}] \leavevmode\begin{itemize}
\item {} 
\sphinxstyleliteralstrong{\sphinxupquote{s}} \textendash{} numpy ndarray of shape (M,N) of N functions with M samples

\item {} 
\sphinxstyleliteralstrong{\sphinxupquote{b}} \textendash{} numpy ndarray of shape (M,N) of N functions with M samples

\item {} 
\sphinxstyleliteralstrong{\sphinxupquote{X}} \textendash{} numpy ndarray of shape (M,N) of N functions with M samples

\item {} 
\sphinxstyleliteralstrong{\sphinxupquote{y}} \textendash{} numpy ndarray of shape (1,N) responses

\end{itemize}

\item[{Return type}] \leavevmode
numpy array

\item[{Return out}] \leavevmode
hessian of logistic loss

\end{description}\end{quote}

\end{fulllineitems}

\index{logit\_loss() (in module regression)@\spxentry{logit\_loss()}\spxextra{in module regression}}

\begin{fulllineitems}
\phantomsection\label{\detokenize{regression:regression.logit_loss}}\pysiglinewithargsret{\sphinxcode{\sphinxupquote{regression.}}\sphinxbfcode{\sphinxupquote{logit\_loss}}}{\emph{b}, \emph{X}, \emph{y}}{}
logistic loss function, returns Sum\{-log(phi(t))\}
\begin{quote}\begin{description}
\item[{Parameters}] \leavevmode\begin{itemize}
\item {} 
\sphinxstyleliteralstrong{\sphinxupquote{b}} \textendash{} numpy ndarray of shape (M,N) of N functions with M samples

\item {} 
\sphinxstyleliteralstrong{\sphinxupquote{X}} \textendash{} numpy ndarray of shape (M,N) of N functions with M samples

\item {} 
\sphinxstyleliteralstrong{\sphinxupquote{y}} \textendash{} numpy ndarray of shape (1,N) of N responses

\end{itemize}

\item[{Return type}] \leavevmode
numpy array

\item[{Return out}] \leavevmode
loss value

\end{description}\end{quote}

\end{fulllineitems}

\index{mlogit\_gradient() (in module regression)@\spxentry{mlogit\_gradient()}\spxextra{in module regression}}

\begin{fulllineitems}
\phantomsection\label{\detokenize{regression:regression.mlogit_gradient}}\pysiglinewithargsret{\sphinxcode{\sphinxupquote{regression.}}\sphinxbfcode{\sphinxupquote{mlogit\_gradient}}}{\emph{b}, \emph{X}, \emph{Y}}{}
calculates gradient of the multinomial logistic loss
\begin{quote}\begin{description}
\item[{Parameters}] \leavevmode\begin{itemize}
\item {} 
\sphinxstyleliteralstrong{\sphinxupquote{b}} \textendash{} numpy ndarray of shape (M,N) of N functions with M samples

\item {} 
\sphinxstyleliteralstrong{\sphinxupquote{X}} \textendash{} numpy ndarray of shape (M,N) of N functions with M samples

\item {} 
\sphinxstyleliteralstrong{\sphinxupquote{y}} \textendash{} numpy ndarray of shape (1,N) responses

\end{itemize}

\item[{Return type}] \leavevmode
numpy array

\item[{Return grad}] \leavevmode
gradient

\end{description}\end{quote}

\end{fulllineitems}

\index{mlogit\_loss() (in module regression)@\spxentry{mlogit\_loss()}\spxextra{in module regression}}

\begin{fulllineitems}
\phantomsection\label{\detokenize{regression:regression.mlogit_loss}}\pysiglinewithargsret{\sphinxcode{\sphinxupquote{regression.}}\sphinxbfcode{\sphinxupquote{mlogit\_loss}}}{\emph{b}, \emph{X}, \emph{Y}}{}
calculates multinomial logistic loss (negative log-likelihood)
\begin{quote}\begin{description}
\item[{Parameters}] \leavevmode\begin{itemize}
\item {} 
\sphinxstyleliteralstrong{\sphinxupquote{b}} \textendash{} numpy ndarray of shape (M,N) of N functions with M samples

\item {} 
\sphinxstyleliteralstrong{\sphinxupquote{X}} \textendash{} numpy ndarray of shape (M,N) of N functions with M samples

\item {} 
\sphinxstyleliteralstrong{\sphinxupquote{y}} \textendash{} numpy ndarray of shape (1,N) responses

\end{itemize}

\item[{Return type}] \leavevmode
numpy array

\item[{Return nll}] \leavevmode
negative log-likelihood

\end{description}\end{quote}

\end{fulllineitems}

\index{mlogit\_warp\_grad() (in module regression)@\spxentry{mlogit\_warp\_grad()}\spxextra{in module regression}}

\begin{fulllineitems}
\phantomsection\label{\detokenize{regression:regression.mlogit_warp_grad}}\pysiglinewithargsret{\sphinxcode{\sphinxupquote{regression.}}\sphinxbfcode{\sphinxupquote{mlogit\_warp\_grad}}}{\emph{alpha}, \emph{beta}, \emph{time}, \emph{q}, \emph{y}, \emph{max\_itr=8000}, \emph{tol=1e-10}, \emph{delta=0.008}, \emph{display=0}}{}
calculates optimal warping for functional multinomial logistic regression
\begin{quote}\begin{description}
\item[{Parameters}] \leavevmode\begin{itemize}
\item {} 
\sphinxstyleliteralstrong{\sphinxupquote{alpha}} \textendash{} scalar

\item {} 
\sphinxstyleliteralstrong{\sphinxupquote{beta}} \textendash{} numpy ndarray of shape (M,N) of N functions with M samples

\item {} 
\sphinxstyleliteralstrong{\sphinxupquote{time}} \textendash{} vector of size M describing the sample points

\item {} 
\sphinxstyleliteralstrong{\sphinxupquote{q}} \textendash{} numpy ndarray of shape (M,N) of N functions with M samples

\item {} 
\sphinxstyleliteralstrong{\sphinxupquote{y}} \textendash{} numpy ndarray of shape (1,N) responses

\item {} 
\sphinxstyleliteralstrong{\sphinxupquote{max\_itr}} \textendash{} maximum number of iterations (Default=8000)

\item {} 
\sphinxstyleliteralstrong{\sphinxupquote{tol}} \textendash{} stopping tolerance (Default=1e-10)

\item {} 
\sphinxstyleliteralstrong{\sphinxupquote{delta}} \textendash{} gradient step size (Default=0.008)

\item {} 
\sphinxstyleliteralstrong{\sphinxupquote{display}} \textendash{} display iterations (Default=0)

\end{itemize}

\item[{Return type}] \leavevmode
tuple of numpy array

\item[{Return gam\_old}] \leavevmode
warping function

\end{description}\end{quote}

\end{fulllineitems}

\index{phi() (in module regression)@\spxentry{phi()}\spxextra{in module regression}}

\begin{fulllineitems}
\phantomsection\label{\detokenize{regression:regression.phi}}\pysiglinewithargsret{\sphinxcode{\sphinxupquote{regression.}}\sphinxbfcode{\sphinxupquote{phi}}}{\emph{t}}{}
calculates logistic function, returns 1 / (1 + exp(-t))
\begin{quote}\begin{description}
\item[{Parameters}] \leavevmode
\sphinxstyleliteralstrong{\sphinxupquote{t}} \textendash{} scalar

\item[{Return type}] \leavevmode
numpy array

\item[{Return out}] \leavevmode
return value

\end{description}\end{quote}

\end{fulllineitems}

\index{regression\_warp() (in module regression)@\spxentry{regression\_warp()}\spxextra{in module regression}}

\begin{fulllineitems}
\phantomsection\label{\detokenize{regression:regression.regression_warp}}\pysiglinewithargsret{\sphinxcode{\sphinxupquote{regression.}}\sphinxbfcode{\sphinxupquote{regression\_warp}}}{\emph{beta}, \emph{time}, \emph{q}, \emph{y}, \emph{alpha}}{}
calculates optimal warping for function linear regression
\begin{quote}\begin{description}
\item[{Parameters}] \leavevmode\begin{itemize}
\item {} 
\sphinxstyleliteralstrong{\sphinxupquote{beta}} \textendash{} numpy ndarray of shape (M,N) of M functions with N samples

\item {} 
\sphinxstyleliteralstrong{\sphinxupquote{time}} \textendash{} vector of size N describing the sample points

\item {} 
\sphinxstyleliteralstrong{\sphinxupquote{q}} \textendash{} numpy ndarray of shape (M,N) of M functions with N samples

\item {} 
\sphinxstyleliteralstrong{\sphinxupquote{y}} \textendash{} numpy ndarray of shape (1,N) of M functions with N samples

\end{itemize}

\end{description}\end{quote}

responses
:param alpha: numpy scalar
\begin{quote}\begin{description}
\item[{Return type}] \leavevmode
numpy array

\item[{Return gamma\_new}] \leavevmode
warping function

\end{description}\end{quote}

\end{fulllineitems}



\chapter{Elastic Principal Component Regression}
\label{\detokenize{pcr_regression:module-pcr_regression}}\label{\detokenize{pcr_regression:elastic-principal-component-regression}}\label{\detokenize{pcr_regression::doc}}\index{pcr\_regression (module)@\spxentry{pcr\_regression}\spxextra{module}}
Warping PCR Invariant Regression using SRSF

moduleauthor:: Derek Tucker \textless{}\sphinxhref{mailto:jdtuck@sandia.gov}{jdtuck@sandia.gov}\textgreater{}
\index{elastic\_lpcr\_regression() (in module pcr\_regression)@\spxentry{elastic\_lpcr\_regression()}\spxextra{in module pcr\_regression}}

\begin{fulllineitems}
\phantomsection\label{\detokenize{pcr_regression:pcr_regression.elastic_lpcr_regression}}\pysiglinewithargsret{\sphinxcode{\sphinxupquote{pcr\_regression.}}\sphinxbfcode{\sphinxupquote{elastic\_lpcr\_regression}}}{\emph{f}, \emph{y}, \emph{time}, \emph{pca\_method='combined'}, \emph{no=5}, \emph{smooth\_data=False}, \emph{sparam=25}}{}
This function identifies a logistic regression model with phase-variability
using elastic pca
\begin{quote}\begin{description}
\item[{Parameters}] \leavevmode\begin{itemize}
\item {} 
\sphinxstyleliteralstrong{\sphinxupquote{f}} (\sphinxstyleliteralemphasis{\sphinxupquote{np.ndarray}}) \textendash{} numpy ndarray of shape (M,N) of N functions with M samples

\item {} 
\sphinxstyleliteralstrong{\sphinxupquote{y}} \textendash{} numpy array of N responses

\item {} 
\sphinxstyleliteralstrong{\sphinxupquote{time}} (\sphinxstyleliteralemphasis{\sphinxupquote{np.ndarray}}) \textendash{} vector of size M describing the sample points

\item {} 
\sphinxstyleliteralstrong{\sphinxupquote{pca\_method}} \textendash{} string specifing pca method (options = “combined”,
“vert”, or “horiz”, default = “combined”)

\item {} 
\sphinxstyleliteralstrong{\sphinxupquote{no}} \textendash{} scalar specify number of principal components (default=5)

\item {} 
\sphinxstyleliteralstrong{\sphinxupquote{smooth\_data}} \textendash{} smooth data using box filter (default = F)

\item {} 
\sphinxstyleliteralstrong{\sphinxupquote{sparam}} \textendash{} number of times to apply box filter (default = 25)

\end{itemize}

\item[{Return type}] \leavevmode
tuple of numpy array

\item[{Return alpha}] \leavevmode
alpha parameter of model

\item[{Return b}] \leavevmode
regressor vector

\item[{Return y}] \leavevmode
response vector

\item[{Return warp\_data}] \leavevmode
alignment object from srsf\_align

\item[{Return pca}] \leavevmode
fpca object from corresponding pca method

\item[{Return Loss}] \leavevmode
logistic loss

\item[{Return pca.method}] \leavevmode
string of pca method

\end{description}\end{quote}

\end{fulllineitems}

\index{elastic\_mlpcr\_regression() (in module pcr\_regression)@\spxentry{elastic\_mlpcr\_regression()}\spxextra{in module pcr\_regression}}

\begin{fulllineitems}
\phantomsection\label{\detokenize{pcr_regression:pcr_regression.elastic_mlpcr_regression}}\pysiglinewithargsret{\sphinxcode{\sphinxupquote{pcr\_regression.}}\sphinxbfcode{\sphinxupquote{elastic\_mlpcr\_regression}}}{\emph{f}, \emph{y}, \emph{time}, \emph{pca\_method='combined'}, \emph{no=5}, \emph{smooth\_data=False}, \emph{sparam=25}}{}
This function identifies a logistic regression model with phase-variability
using elastic pca
\begin{quote}\begin{description}
\item[{Parameters}] \leavevmode\begin{itemize}
\item {} 
\sphinxstyleliteralstrong{\sphinxupquote{f}} (\sphinxstyleliteralemphasis{\sphinxupquote{np.ndarray}}) \textendash{} numpy ndarray of shape (M,N) of N functions with M samples

\item {} 
\sphinxstyleliteralstrong{\sphinxupquote{y}} \textendash{} numpy array of N responses

\item {} 
\sphinxstyleliteralstrong{\sphinxupquote{time}} (\sphinxstyleliteralemphasis{\sphinxupquote{np.ndarray}}) \textendash{} vector of size M describing the sample points

\item {} 
\sphinxstyleliteralstrong{\sphinxupquote{pca\_method}} \textendash{} string specifing pca method (options = “combined”,
“vert”, or “horiz”, default = “combined”)

\item {} 
\sphinxstyleliteralstrong{\sphinxupquote{no}} \textendash{} scalar specify number of principal components (default=5)

\item {} 
\sphinxstyleliteralstrong{\sphinxupquote{smooth\_data}} \textendash{} smooth data using box filter (default = F)

\item {} 
\sphinxstyleliteralstrong{\sphinxupquote{sparam}} \textendash{} number of times to apply box filter (default = 25)

\end{itemize}

\item[{Return type}] \leavevmode
tuple of numpy array

\item[{Return alpha}] \leavevmode
alpha parameter of model

\item[{Return b}] \leavevmode
regressor vector

\item[{Return y}] \leavevmode
response vector

\item[{Return warp\_data}] \leavevmode
alignment object from srsf\_align

\item[{Return pca}] \leavevmode
fpca object from corresponding pca method

\item[{Return Loss}] \leavevmode
logistic loss

\item[{Return pca.method}] \leavevmode
string of pca method

\end{description}\end{quote}

\end{fulllineitems}

\index{elastic\_pcr\_regression() (in module pcr\_regression)@\spxentry{elastic\_pcr\_regression()}\spxextra{in module pcr\_regression}}

\begin{fulllineitems}
\phantomsection\label{\detokenize{pcr_regression:pcr_regression.elastic_pcr_regression}}\pysiglinewithargsret{\sphinxcode{\sphinxupquote{pcr\_regression.}}\sphinxbfcode{\sphinxupquote{elastic\_pcr\_regression}}}{\emph{f}, \emph{y}, \emph{time}, \emph{pca\_method='combined'}, \emph{no=5}, \emph{smooth\_data=False}, \emph{sparam=25}, \emph{parallel=False}, \emph{C=None}}{}
This function identifies a regression model with phase-variability
using elastic pca
\begin{quote}\begin{description}
\item[{Parameters}] \leavevmode\begin{itemize}
\item {} 
\sphinxstyleliteralstrong{\sphinxupquote{f}} (\sphinxstyleliteralemphasis{\sphinxupquote{np.ndarray}}) \textendash{} numpy ndarray of shape (M,N) of N functions with M samples

\item {} 
\sphinxstyleliteralstrong{\sphinxupquote{y}} \textendash{} numpy array of N responses

\item {} 
\sphinxstyleliteralstrong{\sphinxupquote{time}} (\sphinxstyleliteralemphasis{\sphinxupquote{np.ndarray}}) \textendash{} vector of size M describing the sample points

\item {} 
\sphinxstyleliteralstrong{\sphinxupquote{pca\_method}} \textendash{} string specifing pca method (options = “combined”,
“vert”, or “horiz”, default = “combined”)

\item {} 
\sphinxstyleliteralstrong{\sphinxupquote{no}} \textendash{} scalar specify number of principal components (default=5)

\item {} 
\sphinxstyleliteralstrong{\sphinxupquote{smooth\_data}} \textendash{} smooth data using box filter (default = F)

\item {} 
\sphinxstyleliteralstrong{\sphinxupquote{sparam}} \textendash{} number of times to apply box filter (default = 25)

\item {} 
\sphinxstyleliteralstrong{\sphinxupquote{parallel}} \textendash{} run in parallel (default = F)

\item {} 
\sphinxstyleliteralstrong{\sphinxupquote{C}} \textendash{} scale balance parameter for combined method (default = None)

\end{itemize}

\item[{Return type}] \leavevmode
tuple of numpy array

\item[{Return alpha}] \leavevmode
alpha parameter of model

\item[{Return b}] \leavevmode
regressor vector

\item[{Return y}] \leavevmode
response vector

\item[{Return warp\_data}] \leavevmode
alignment object from srsf\_align

\item[{Return pca}] \leavevmode
fpca object from corresponding pca method

\item[{Return SSE}] \leavevmode
sum of squared errors

\item[{Return pca.method}] \leavevmode
string of pca method

\end{description}\end{quote}

\end{fulllineitems}



\chapter{Elastic Functional Tolerance Bounds}
\label{\detokenize{tolerance:module-tolerance}}\label{\detokenize{tolerance:elastic-functional-tolerance-bounds}}\label{\detokenize{tolerance::doc}}\index{tolerance (module)@\spxentry{tolerance}\spxextra{module}}
Functional Tolerance Bounds using SRSF

moduleauthor:: Derek Tucker \textless{}\sphinxhref{mailto:jdtuck@sandia.gov}{jdtuck@sandia.gov}\textgreater{}
\index{bootTB() (in module tolerance)@\spxentry{bootTB()}\spxextra{in module tolerance}}

\begin{fulllineitems}
\phantomsection\label{\detokenize{tolerance:tolerance.bootTB}}\pysiglinewithargsret{\sphinxcode{\sphinxupquote{tolerance.}}\sphinxbfcode{\sphinxupquote{bootTB}}}{\emph{f}, \emph{time}, \emph{a=0.5}, \emph{p=0.99}, \emph{B=500}, \emph{no=5}, \emph{parallel=True}}{}
This function computes tolerance bounds for functional data containing
phase and amplitude variation using bootstrap sampling
\begin{quote}\begin{description}
\item[{Parameters}] \leavevmode\begin{itemize}
\item {} 
\sphinxstyleliteralstrong{\sphinxupquote{f}} (\sphinxstyleliteralemphasis{\sphinxupquote{np.ndarray}}) \textendash{} numpy ndarray of shape (M,N) of N functions with M samples

\item {} 
\sphinxstyleliteralstrong{\sphinxupquote{time}} (\sphinxstyleliteralemphasis{\sphinxupquote{np.ndarray}}) \textendash{} vector of size M describing the sample points

\item {} 
\sphinxstyleliteralstrong{\sphinxupquote{a}} \textendash{} confidence level of tolerance bound (default = 0.05)

\item {} 
\sphinxstyleliteralstrong{\sphinxupquote{p}} \textendash{} coverage level of tolerance bound (default = 0.99)

\item {} 
\sphinxstyleliteralstrong{\sphinxupquote{B}} \textendash{} number of bootstrap samples (default = 500)

\item {} 
\sphinxstyleliteralstrong{\sphinxupquote{no}} \textendash{} number of principal components (default = 5)

\item {} 
\sphinxstyleliteralstrong{\sphinxupquote{parallel}} \textendash{} enable parallel processing (default = T)

\end{itemize}

\item[{Return type}] \leavevmode
tuple of boxplot objects

\item[{Return amp}] \leavevmode
amplitude tolerance bounds

\item[{Return ph}] \leavevmode
phase tolerance bounds

\end{description}\end{quote}

\end{fulllineitems}

\index{mvtol\_region() (in module tolerance)@\spxentry{mvtol\_region()}\spxextra{in module tolerance}}

\begin{fulllineitems}
\phantomsection\label{\detokenize{tolerance:tolerance.mvtol_region}}\pysiglinewithargsret{\sphinxcode{\sphinxupquote{tolerance.}}\sphinxbfcode{\sphinxupquote{mvtol\_region}}}{\emph{x}, \emph{alpha}, \emph{P}, \emph{B}}{}
Computes tolerance factor for multivariate normal

Krishnamoorthy, K. and Mondal, S. (2006), Improved Tolerance Factors for Multivariate Normal
Distributions, Communications in Statistics - Simulation and Computation, 35, 461\textendash{}478.
\begin{quote}\begin{description}
\item[{Parameters}] \leavevmode\begin{itemize}
\item {} 
\sphinxstyleliteralstrong{\sphinxupquote{x}} \textendash{} (M,N) matrix defining N variables of M samples

\item {} 
\sphinxstyleliteralstrong{\sphinxupquote{alpha}} \textendash{} confidence level

\item {} 
\sphinxstyleliteralstrong{\sphinxupquote{P}} \textendash{} coverage level

\item {} 
\sphinxstyleliteralstrong{\sphinxupquote{B}} \textendash{} number of bootstrap samples

\end{itemize}

\item[{Return type}] \leavevmode
double

\item[{Return tol}] \leavevmode
tolerance factor

\end{description}\end{quote}

\end{fulllineitems}

\index{pcaTB() (in module tolerance)@\spxentry{pcaTB()}\spxextra{in module tolerance}}

\begin{fulllineitems}
\phantomsection\label{\detokenize{tolerance:tolerance.pcaTB}}\pysiglinewithargsret{\sphinxcode{\sphinxupquote{tolerance.}}\sphinxbfcode{\sphinxupquote{pcaTB}}}{\emph{f}, \emph{time}, \emph{a=0.5}, \emph{p=0.99}, \emph{no=5}, \emph{parallel=True}}{}
This function computes tolerance bounds for functional data containing
phase and amplitude variation using fPCA
\begin{quote}\begin{description}
\item[{Parameters}] \leavevmode\begin{itemize}
\item {} 
\sphinxstyleliteralstrong{\sphinxupquote{f}} (\sphinxstyleliteralemphasis{\sphinxupquote{np.ndarray}}) \textendash{} numpy ndarray of shape (M,N) of N functions with M samples

\item {} 
\sphinxstyleliteralstrong{\sphinxupquote{time}} (\sphinxstyleliteralemphasis{\sphinxupquote{np.ndarray}}) \textendash{} vector of size M describing the sample points

\item {} 
\sphinxstyleliteralstrong{\sphinxupquote{a}} \textendash{} confidence level of tolerance bound (default = 0.05)

\item {} 
\sphinxstyleliteralstrong{\sphinxupquote{p}} \textendash{} coverage level of tolerance bound (default = 0.99)

\item {} 
\sphinxstyleliteralstrong{\sphinxupquote{no}} \textendash{} number of principal components (default = 5)

\item {} 
\sphinxstyleliteralstrong{\sphinxupquote{parallel}} \textendash{} enable parallel processing (default = T)

\end{itemize}

\item[{Return type}] \leavevmode
tuple of boxplot objects

\item[{Return warp}] \leavevmode
alignment data from time\_warping

\item[{Return pca}] \leavevmode
functional pca from jointFPCA

\item[{Return tol}] \leavevmode
tolerance factor

\end{description}\end{quote}

\end{fulllineitems}

\index{rwishart() (in module tolerance)@\spxentry{rwishart()}\spxextra{in module tolerance}}

\begin{fulllineitems}
\phantomsection\label{\detokenize{tolerance:tolerance.rwishart}}\pysiglinewithargsret{\sphinxcode{\sphinxupquote{tolerance.}}\sphinxbfcode{\sphinxupquote{rwishart}}}{\emph{df}, \emph{p}}{}
Computes a random wishart matrix
\begin{quote}\begin{description}
\item[{Parameters}] \leavevmode\begin{itemize}
\item {} 
\sphinxstyleliteralstrong{\sphinxupquote{df}} \textendash{} degree of freedom

\item {} 
\sphinxstyleliteralstrong{\sphinxupquote{p}} \textendash{} number of dimensions

\end{itemize}

\item[{Return type}] \leavevmode
double

\item[{Return R}] \leavevmode
matrix

\end{description}\end{quote}

\end{fulllineitems}



\chapter{SRVF Geodesic Computation}
\label{\detokenize{geodesic:module-geodesic}}\label{\detokenize{geodesic:srvf-geodesic-computation}}\label{\detokenize{geodesic::doc}}\index{geodesic (module)@\spxentry{geodesic}\spxextra{module}}
geodesic calculation for SRVF (curves) open and closed)

moduleauthor:: Derek Tucker \textless{}\sphinxhref{mailto:jdtuck@sandia.gov}{jdtuck@sandia.gov}\textgreater{}
\index{back\_parallel\_transport() (in module geodesic)@\spxentry{back\_parallel\_transport()}\spxextra{in module geodesic}}

\begin{fulllineitems}
\phantomsection\label{\detokenize{geodesic:geodesic.back_parallel_transport}}\pysiglinewithargsret{\sphinxcode{\sphinxupquote{geodesic.}}\sphinxbfcode{\sphinxupquote{back\_parallel\_transport}}}{\emph{u1}, \emph{alpha}, \emph{basis}, \emph{T=100}, \emph{k=5}}{}
backwards parallel translates q1 and q2 along manifold
\begin{quote}\begin{description}
\item[{Parameters}] \leavevmode\begin{itemize}
\item {} 
\sphinxstyleliteralstrong{\sphinxupquote{u1}} \textendash{} numpy ndarray of shape (2,M) of M samples

\item {} 
\sphinxstyleliteralstrong{\sphinxupquote{alpha}} \textendash{} numpy ndarray of shape (2,M) of M samples

\item {} 
\sphinxstyleliteralstrong{\sphinxupquote{basis}} \textendash{} list numpy ndarray of shape (2,M) of M samples

\item {} 
\sphinxstyleliteralstrong{\sphinxupquote{T}} \textendash{} Number of samples of curve (Default = 100)

\item {} 
\sphinxstyleliteralstrong{\sphinxupquote{k}} \textendash{} number of samples along path (Default = 5)

\end{itemize}

\item[{Return type}] \leavevmode
numpy ndarray

\item[{Return utilde}] \leavevmode
translated vector

\end{description}\end{quote}

\end{fulllineitems}

\index{calc\_alphadot() (in module geodesic)@\spxentry{calc\_alphadot()}\spxextra{in module geodesic}}

\begin{fulllineitems}
\phantomsection\label{\detokenize{geodesic:geodesic.calc_alphadot}}\pysiglinewithargsret{\sphinxcode{\sphinxupquote{geodesic.}}\sphinxbfcode{\sphinxupquote{calc\_alphadot}}}{\emph{alpha}, \emph{basis}, \emph{T=100}, \emph{k=5}}{}
calculates derivative along the path alpha
\begin{quote}\begin{description}
\item[{Parameters}] \leavevmode\begin{itemize}
\item {} 
\sphinxstyleliteralstrong{\sphinxupquote{alpha}} \textendash{} numpy ndarray of shape (2,M) of M samples

\item {} 
\sphinxstyleliteralstrong{\sphinxupquote{basis}} \textendash{} list of numpy ndarray of shape (2,M) of M samples

\item {} 
\sphinxstyleliteralstrong{\sphinxupquote{T}} \textendash{} Number of samples of curve (Default = 100)

\item {} 
\sphinxstyleliteralstrong{\sphinxupquote{k}} \textendash{} number of samples along path (Default = 5)

\end{itemize}

\item[{Return type}] \leavevmode
numpy ndarray

\item[{Return alphadot}] \leavevmode
derivative of alpha

\end{description}\end{quote}

\end{fulllineitems}

\index{calculate\_energy() (in module geodesic)@\spxentry{calculate\_energy()}\spxextra{in module geodesic}}

\begin{fulllineitems}
\phantomsection\label{\detokenize{geodesic:geodesic.calculate_energy}}\pysiglinewithargsret{\sphinxcode{\sphinxupquote{geodesic.}}\sphinxbfcode{\sphinxupquote{calculate\_energy}}}{\emph{alphadot}, \emph{T=100}, \emph{k=5}}{}
calculates energy along path
\begin{quote}\begin{description}
\item[{Parameters}] \leavevmode\begin{itemize}
\item {} 
\sphinxstyleliteralstrong{\sphinxupquote{alphadot}} \textendash{} numpy ndarray of shape (2,M) of M samples

\item {} 
\sphinxstyleliteralstrong{\sphinxupquote{T}} \textendash{} Number of samples of curve (Default = 100)

\item {} 
\sphinxstyleliteralstrong{\sphinxupquote{k}} \textendash{} number of samples along path (Default = 5)

\end{itemize}

\item[{Return type}] \leavevmode
numpy scalar

\item[{Return E}] \leavevmode
energy

\end{description}\end{quote}

\end{fulllineitems}

\index{calculate\_gradE() (in module geodesic)@\spxentry{calculate\_gradE()}\spxextra{in module geodesic}}

\begin{fulllineitems}
\phantomsection\label{\detokenize{geodesic:geodesic.calculate_gradE}}\pysiglinewithargsret{\sphinxcode{\sphinxupquote{geodesic.}}\sphinxbfcode{\sphinxupquote{calculate\_gradE}}}{\emph{u}, \emph{utilde}, \emph{T=100}, \emph{k=5}}{}
calculates gradient of energy along path
\begin{quote}\begin{description}
\item[{Parameters}] \leavevmode\begin{itemize}
\item {} 
\sphinxstyleliteralstrong{\sphinxupquote{u}} \textendash{} numpy ndarray of shape (2,M) of M samples

\item {} 
\sphinxstyleliteralstrong{\sphinxupquote{utilde}} \textendash{} numpy ndarray of shape (2,M) of M samples

\item {} 
\sphinxstyleliteralstrong{\sphinxupquote{T}} \textendash{} Number of samples of curve (Default = 100)

\item {} 
\sphinxstyleliteralstrong{\sphinxupquote{k}} \textendash{} number of samples along path (Default = 5)

\end{itemize}

\item[{Return type}] \leavevmode
numpy scalar

\item[{Return gradE}] \leavevmode
gradient of energy

\item[{Return normgradE}] \leavevmode
norm of gradient of energy

\end{description}\end{quote}

\end{fulllineitems}

\index{cov\_integral() (in module geodesic)@\spxentry{cov\_integral()}\spxextra{in module geodesic}}

\begin{fulllineitems}
\phantomsection\label{\detokenize{geodesic:geodesic.cov_integral}}\pysiglinewithargsret{\sphinxcode{\sphinxupquote{geodesic.}}\sphinxbfcode{\sphinxupquote{cov\_integral}}}{\emph{alpha}, \emph{alphadot}, \emph{basis}, \emph{T=100}, \emph{k=5}}{}
Calculates covariance along path alpha
\begin{quote}\begin{description}
\item[{Parameters}] \leavevmode\begin{itemize}
\item {} 
\sphinxstyleliteralstrong{\sphinxupquote{alpha}} \textendash{} numpy ndarray of shape (2,M) of M samples (first curve)

\item {} 
\sphinxstyleliteralstrong{\sphinxupquote{alphadot}} \textendash{} numpy ndarray of shape (2,M) of M samples

\item {} 
\sphinxstyleliteralstrong{\sphinxupquote{basis}} \textendash{} list numpy ndarray of shape (2,M) of M samples

\item {} 
\sphinxstyleliteralstrong{\sphinxupquote{T}} \textendash{} Number of samples of curve (Default = 100)

\item {} 
\sphinxstyleliteralstrong{\sphinxupquote{k}} \textendash{} number of samples along path (Default = 5)

\end{itemize}

\item[{Return type}] \leavevmode
numpy ndarray

\item[{Return u}] \leavevmode
covariance

\end{description}\end{quote}

\end{fulllineitems}

\index{find\_basis\_normal\_path() (in module geodesic)@\spxentry{find\_basis\_normal\_path()}\spxextra{in module geodesic}}

\begin{fulllineitems}
\phantomsection\label{\detokenize{geodesic:geodesic.find_basis_normal_path}}\pysiglinewithargsret{\sphinxcode{\sphinxupquote{geodesic.}}\sphinxbfcode{\sphinxupquote{find\_basis\_normal\_path}}}{\emph{alpha}, \emph{k=5}}{}
computes orthonormalized basis vectors to the normal space at each of the
k points (q-functions) of the path alpha
\begin{quote}\begin{description}
\item[{Parameters}] \leavevmode\begin{itemize}
\item {} 
\sphinxstyleliteralstrong{\sphinxupquote{alpha}} \textendash{} numpy ndarray of shape (2,M) of M samples (path)

\item {} 
\sphinxstyleliteralstrong{\sphinxupquote{k}} \textendash{} number of samples along path (Default = 5)

\end{itemize}

\item[{Return type}] \leavevmode
numpy ndarray

\item[{Return basis}] \leavevmode
basis vectors along the path

\end{description}\end{quote}

\end{fulllineitems}

\index{geod\_dist\_path\_strt() (in module geodesic)@\spxentry{geod\_dist\_path\_strt()}\spxextra{in module geodesic}}

\begin{fulllineitems}
\phantomsection\label{\detokenize{geodesic:geodesic.geod_dist_path_strt}}\pysiglinewithargsret{\sphinxcode{\sphinxupquote{geodesic.}}\sphinxbfcode{\sphinxupquote{geod\_dist\_path\_strt}}}{\emph{beta}, \emph{k=5}}{}
calculate geodisc distance for path straightening
\begin{quote}\begin{description}
\item[{Parameters}] \leavevmode\begin{itemize}
\item {} 
\sphinxstyleliteralstrong{\sphinxupquote{beta}} \textendash{} numpy ndarray of shape (2,M) of M samples

\item {} 
\sphinxstyleliteralstrong{\sphinxupquote{k}} \textendash{} number of samples along path (Default = 5)

\end{itemize}

\item[{Return type}] \leavevmode
numpy scalar

\item[{Return dist}] \leavevmode
geodesic distance

\end{description}\end{quote}

\end{fulllineitems}

\index{geod\_sphere() (in module geodesic)@\spxentry{geod\_sphere()}\spxextra{in module geodesic}}

\begin{fulllineitems}
\phantomsection\label{\detokenize{geodesic:geodesic.geod_sphere}}\pysiglinewithargsret{\sphinxcode{\sphinxupquote{geodesic.}}\sphinxbfcode{\sphinxupquote{geod\_sphere}}}{\emph{beta1}, \emph{beta2}, \emph{k=5}}{}
This function caluclates the geodecis between open curves beta1 and
beta2 with k steps along path
\begin{quote}\begin{description}
\item[{Parameters}] \leavevmode\begin{itemize}
\item {} 
\sphinxstyleliteralstrong{\sphinxupquote{beta1}} \textendash{} numpy ndarray of shape (2,M) of M samples

\item {} 
\sphinxstyleliteralstrong{\sphinxupquote{beta2}} \textendash{} numpy ndarray of shape (2,M) of M samples

\item {} 
\sphinxstyleliteralstrong{\sphinxupquote{k}} \textendash{} number of samples along path (Default = 5)

\end{itemize}

\item[{Return type}] \leavevmode
numpy ndarray

\item[{Return dist}] \leavevmode
geodesic distance

\item[{Return path}] \leavevmode
geodesic path

\item[{Return O}] \leavevmode
rotation matrix

\end{description}\end{quote}

\end{fulllineitems}

\index{init\_path\_geod() (in module geodesic)@\spxentry{init\_path\_geod()}\spxextra{in module geodesic}}

\begin{fulllineitems}
\phantomsection\label{\detokenize{geodesic:geodesic.init_path_geod}}\pysiglinewithargsret{\sphinxcode{\sphinxupquote{geodesic.}}\sphinxbfcode{\sphinxupquote{init\_path\_geod}}}{\emph{beta1}, \emph{beta2}, \emph{T=100}, \emph{k=5}}{}
Initializes a path in cal\{C\}. beta1, beta2 are already
standardized curves. Creates a path from beta1 to beta2 in
shape space, then projects to the closed shape manifold.
\begin{quote}\begin{description}
\item[{Parameters}] \leavevmode\begin{itemize}
\item {} 
\sphinxstyleliteralstrong{\sphinxupquote{beta1}} \textendash{} numpy ndarray of shape (2,M) of M samples (first curve)

\item {} 
\sphinxstyleliteralstrong{\sphinxupquote{beta2}} \textendash{} numpy ndarray of shape (2,M) of M samples (end curve)

\item {} 
\sphinxstyleliteralstrong{\sphinxupquote{T}} \textendash{} Number of samples of curve (Default = 100)

\item {} 
\sphinxstyleliteralstrong{\sphinxupquote{k}} \textendash{} number of samples along path (Default = 5)

\end{itemize}

\item[{Return type}] \leavevmode
numpy ndarray

\item[{Return alpha}] \leavevmode
a path between two q-functions

\item[{Return beta}] \leavevmode
a path between two curves

\item[{Return O}] \leavevmode
rotation matrix

\end{description}\end{quote}

\end{fulllineitems}

\index{init\_path\_rand() (in module geodesic)@\spxentry{init\_path\_rand()}\spxextra{in module geodesic}}

\begin{fulllineitems}
\phantomsection\label{\detokenize{geodesic:geodesic.init_path_rand}}\pysiglinewithargsret{\sphinxcode{\sphinxupquote{geodesic.}}\sphinxbfcode{\sphinxupquote{init\_path\_rand}}}{\emph{beta1}, \emph{beta\_mid}, \emph{beta2}, \emph{T=100}, \emph{k=5}}{}
Initializes a path in cal\{C\}. beta1, beta\_mid beta2 are already
standardized curves. Creates a path from beta1 to beta\_mid to beta2 in
shape space, then projects to the closed shape manifold.
\begin{quote}\begin{description}
\item[{Parameters}] \leavevmode\begin{itemize}
\item {} 
\sphinxstyleliteralstrong{\sphinxupquote{beta1}} \textendash{} numpy ndarray of shape (2,M) of M samples (first curve)

\item {} 
\sphinxstyleliteralstrong{\sphinxupquote{betamid}} \textendash{} numpy ndarray of shape (2,M) of M samples (mid curve)

\item {} 
\sphinxstyleliteralstrong{\sphinxupquote{beta2}} \textendash{} numpy ndarray of shape (2,M) of M samples (end curve)

\item {} 
\sphinxstyleliteralstrong{\sphinxupquote{T}} \textendash{} Number of samples of curve (Default = 100)

\item {} 
\sphinxstyleliteralstrong{\sphinxupquote{k}} \textendash{} number of samples along path (Default = 5)

\end{itemize}

\item[{Return type}] \leavevmode
numpy ndarray

\item[{Return alpha}] \leavevmode
a path between two q-functions

\item[{Return beta}] \leavevmode
a path between two curves

\item[{Return O}] \leavevmode
rotation matrix

\end{description}\end{quote}

\end{fulllineitems}

\index{path\_straightening() (in module geodesic)@\spxentry{path\_straightening()}\spxextra{in module geodesic}}

\begin{fulllineitems}
\phantomsection\label{\detokenize{geodesic:geodesic.path_straightening}}\pysiglinewithargsret{\sphinxcode{\sphinxupquote{geodesic.}}\sphinxbfcode{\sphinxupquote{path\_straightening}}}{\emph{beta1}, \emph{beta2}, \emph{betamid}, \emph{init='rand'}, \emph{T=100}, \emph{k=5}}{}
Perform path straigtening to find geodesic between two shapes in either
the space of closed curves or the space of affine standardized curves.
This algorithm follows the steps outlined in section 4.6 of the
manuscript.
\begin{quote}\begin{description}
\item[{Parameters}] \leavevmode\begin{itemize}
\item {} 
\sphinxstyleliteralstrong{\sphinxupquote{beta1}} \textendash{} numpy ndarray of shape (2,M) of M samples (first curve)

\item {} 
\sphinxstyleliteralstrong{\sphinxupquote{beta2}} \textendash{} numpy ndarray of shape (2,M) of M samples (end curve)

\item {} 
\sphinxstyleliteralstrong{\sphinxupquote{betamid}} \textendash{} numpy ndarray of shape (2,M) of M samples (mid curve
Default = NULL, only needed for init “rand”)

\item {} 
\sphinxstyleliteralstrong{\sphinxupquote{init}} \textendash{} initilizae path geodesic or random (Default = “rand”)

\item {} 
\sphinxstyleliteralstrong{\sphinxupquote{T}} \textendash{} Number of samples of curve (Default = 100)

\item {} 
\sphinxstyleliteralstrong{\sphinxupquote{k}} \textendash{} number of samples along path (Default = 5)

\end{itemize}

\item[{Return type}] \leavevmode
numpy ndarray

\item[{Return dist}] \leavevmode
geodesic distance

\item[{Return path}] \leavevmode
geodesic path

\item[{Return pathsqnc}] \leavevmode
geodesic path sequence

\item[{Return E}] \leavevmode
energy

\end{description}\end{quote}

\end{fulllineitems}

\index{update\_path() (in module geodesic)@\spxentry{update\_path()}\spxextra{in module geodesic}}

\begin{fulllineitems}
\phantomsection\label{\detokenize{geodesic:geodesic.update_path}}\pysiglinewithargsret{\sphinxcode{\sphinxupquote{geodesic.}}\sphinxbfcode{\sphinxupquote{update\_path}}}{\emph{alpha}, \emph{beta}, \emph{gradE}, \emph{delta}, \emph{T=100}, \emph{k=5}}{}
Update the path along the direction -gradE
\begin{quote}\begin{description}
\item[{Parameters}] \leavevmode\begin{itemize}
\item {} 
\sphinxstyleliteralstrong{\sphinxupquote{alpha}} \textendash{} numpy ndarray of shape (2,M) of M samples

\item {} 
\sphinxstyleliteralstrong{\sphinxupquote{beta}} \textendash{} numpy ndarray of shape (2,M) of M samples

\item {} 
\sphinxstyleliteralstrong{\sphinxupquote{gradE}} \textendash{} numpy ndarray of shape (2,M) of M samples

\item {} 
\sphinxstyleliteralstrong{\sphinxupquote{delta}} \textendash{} gradient paramenter

\item {} 
\sphinxstyleliteralstrong{\sphinxupquote{T}} \textendash{} Number of samples of curve (Default = 100)

\item {} 
\sphinxstyleliteralstrong{\sphinxupquote{k}} \textendash{} number of samples along path (Default = 5)

\end{itemize}

\item[{Return type}] \leavevmode
numpy scalar

\item[{Return alpha}] \leavevmode
updated path of srvfs

\item[{Return beta}] \leavevmode
updated path of curves

\end{description}\end{quote}

\end{fulllineitems}



\chapter{Utility Functions}
\label{\detokenize{utility_functions:module-utility_functions}}\label{\detokenize{utility_functions:utility-functions}}\label{\detokenize{utility_functions::doc}}\index{utility\_functions (module)@\spxentry{utility\_functions}\spxextra{module}}
Utility functions for SRSF Manipulations

moduleauthor:: Derek Tucker \textless{}\sphinxhref{mailto:jdtuck@sandia.gov}{jdtuck@sandia.gov}\textgreater{}
\index{SqrtMean() (in module utility\_functions)@\spxentry{SqrtMean()}\spxextra{in module utility\_functions}}

\begin{fulllineitems}
\phantomsection\label{\detokenize{utility_functions:utility_functions.SqrtMean}}\pysiglinewithargsret{\sphinxcode{\sphinxupquote{utility\_functions.}}\sphinxbfcode{\sphinxupquote{SqrtMean}}}{\emph{gam}}{}
calculates the srsf of warping functions with corresponding shooting vectors
\begin{quote}\begin{description}
\item[{Parameters}] \leavevmode
\sphinxstyleliteralstrong{\sphinxupquote{gam}} \textendash{} numpy ndarray of shape (M,N) of M warping functions
with N samples

\item[{Return type}] \leavevmode
2 numpy ndarray and vector

\item[{Return mu}] \leavevmode
Karcher mean psi function

\item[{Return gam\_mu}] \leavevmode
vector of dim N which is the Karcher mean warping function

\item[{Return psi}] \leavevmode
numpy ndarray of shape (M,N) of M SRSF of the warping functions

\item[{Return vec}] \leavevmode
numpy ndarray of shape (M,N) of M shooting vectors

\end{description}\end{quote}

\end{fulllineitems}

\index{SqrtMeanInverse() (in module utility\_functions)@\spxentry{SqrtMeanInverse()}\spxextra{in module utility\_functions}}

\begin{fulllineitems}
\phantomsection\label{\detokenize{utility_functions:utility_functions.SqrtMeanInverse}}\pysiglinewithargsret{\sphinxcode{\sphinxupquote{utility\_functions.}}\sphinxbfcode{\sphinxupquote{SqrtMeanInverse}}}{\emph{gam}}{}
finds the inverse of the mean of the set of the diffeomorphisms gamma
\begin{quote}\begin{description}
\item[{Parameters}] \leavevmode
\sphinxstyleliteralstrong{\sphinxupquote{gam}} \textendash{} numpy ndarray of shape (M,N) of M warping functions
with N samples

\item[{Return type}] \leavevmode
vector

\item[{Return gamI}] \leavevmode
inverse of gam

\end{description}\end{quote}

\end{fulllineitems}

\index{SqrtMedian() (in module utility\_functions)@\spxentry{SqrtMedian()}\spxextra{in module utility\_functions}}

\begin{fulllineitems}
\phantomsection\label{\detokenize{utility_functions:utility_functions.SqrtMedian}}\pysiglinewithargsret{\sphinxcode{\sphinxupquote{utility\_functions.}}\sphinxbfcode{\sphinxupquote{SqrtMedian}}}{\emph{gam}}{}
calculates the median srsf of warping functions with corresponding shooting vectors
\begin{quote}\begin{description}
\item[{Parameters}] \leavevmode
\sphinxstyleliteralstrong{\sphinxupquote{gam}} \textendash{} numpy ndarray of shape (M,N) of M warping functions
with N samples

\item[{Return type}] \leavevmode
2 numpy ndarray and vector

\item[{Return gam\_median}] \leavevmode
Karcher median warping function

\item[{Return psi\_meidan}] \leavevmode
vector of dim N which is the Karcher median srsf function

\item[{Return psi}] \leavevmode
numpy ndarray of shape (M,N) of M SRSF of the warping functions

\item[{Return vec}] \leavevmode
numpy ndarray of shape (M,N) of M shooting vectors

\end{description}\end{quote}

\end{fulllineitems}

\index{cumtrapzmid() (in module utility\_functions)@\spxentry{cumtrapzmid()}\spxextra{in module utility\_functions}}

\begin{fulllineitems}
\phantomsection\label{\detokenize{utility_functions:utility_functions.cumtrapzmid}}\pysiglinewithargsret{\sphinxcode{\sphinxupquote{utility\_functions.}}\sphinxbfcode{\sphinxupquote{cumtrapzmid}}}{\emph{x}, \emph{y}, \emph{c}, \emph{mid}}{}
cumulative trapezoidal numerical integration taken from midpoint
\begin{quote}\begin{description}
\item[{Parameters}] \leavevmode\begin{itemize}
\item {} 
\sphinxstyleliteralstrong{\sphinxupquote{x}} \textendash{} vector of size N describing the time samples

\item {} 
\sphinxstyleliteralstrong{\sphinxupquote{y}} \textendash{} vector of size N describing the function

\item {} 
\sphinxstyleliteralstrong{\sphinxupquote{c}} \textendash{} midpoint

\item {} 
\sphinxstyleliteralstrong{\sphinxupquote{mid}} \textendash{} midpiont location

\end{itemize}

\item[{Return type}] \leavevmode
vector

\item[{Return fa}] \leavevmode
cumulative integration

\end{description}\end{quote}

\end{fulllineitems}

\index{diffop() (in module utility\_functions)@\spxentry{diffop()}\spxextra{in module utility\_functions}}

\begin{fulllineitems}
\phantomsection\label{\detokenize{utility_functions:utility_functions.diffop}}\pysiglinewithargsret{\sphinxcode{\sphinxupquote{utility\_functions.}}\sphinxbfcode{\sphinxupquote{diffop}}}{\emph{n}, \emph{binsize=1}}{}
Creates a second order differential operator
\begin{quote}\begin{description}
\item[{Parameters}] \leavevmode\begin{itemize}
\item {} 
\sphinxstyleliteralstrong{\sphinxupquote{n}} \textendash{} dimension

\item {} 
\sphinxstyleliteralstrong{\sphinxupquote{binsize}} \textendash{} dx (default = 1)

\end{itemize}

\item[{Return type}] \leavevmode
numpy ndarray

\item[{Return m}] \leavevmode
matrix describing differential operator

\end{description}\end{quote}

\end{fulllineitems}

\index{elastic\_distance() (in module utility\_functions)@\spxentry{elastic\_distance()}\spxextra{in module utility\_functions}}

\begin{fulllineitems}
\phantomsection\label{\detokenize{utility_functions:utility_functions.elastic_distance}}\pysiglinewithargsret{\sphinxcode{\sphinxupquote{utility\_functions.}}\sphinxbfcode{\sphinxupquote{elastic\_distance}}}{\emph{f1}, \emph{f2}, \emph{time}, \emph{lam=0.0}}{}
”
calculates the distances between function, where f1 is aligned to
f2. In other words
calculates the elastic distances
\begin{quote}\begin{description}
\item[{Parameters}] \leavevmode\begin{itemize}
\item {} 
\sphinxstyleliteralstrong{\sphinxupquote{f1}} \textendash{} vector of size N

\item {} 
\sphinxstyleliteralstrong{\sphinxupquote{f2}} \textendash{} vector of size N

\item {} 
\sphinxstyleliteralstrong{\sphinxupquote{time}} \textendash{} vector of size N describing the sample points

\item {} 
\sphinxstyleliteralstrong{\sphinxupquote{lam}} \textendash{} controls the elasticity (default = 0.0)

\end{itemize}

\item[{Return type}] \leavevmode
scalar

\item[{Return Dy}] \leavevmode
amplitude distance

\item[{Return Dx}] \leavevmode
phase distance

\end{description}\end{quote}

\end{fulllineitems}

\index{f\_K\_fold() (in module utility\_functions)@\spxentry{f\_K\_fold()}\spxextra{in module utility\_functions}}

\begin{fulllineitems}
\phantomsection\label{\detokenize{utility_functions:utility_functions.f_K_fold}}\pysiglinewithargsret{\sphinxcode{\sphinxupquote{utility\_functions.}}\sphinxbfcode{\sphinxupquote{f\_K\_fold}}}{\emph{Nobs}, \emph{K=5}}{}
generates sample indices for K-fold cross validation

:param Nobs number of observations
:param K number of folds
\begin{quote}\begin{description}
\item[{Return type}] \leavevmode
numpy ndarray

\item[{Return train}] \leavevmode
train indexes (Nobs*(K-1)/K X K)

\item[{Return test}] \leavevmode
test indexes (Nobs*(1/K) X K)

\end{description}\end{quote}

\end{fulllineitems}

\index{f\_to\_srsf() (in module utility\_functions)@\spxentry{f\_to\_srsf()}\spxextra{in module utility\_functions}}

\begin{fulllineitems}
\phantomsection\label{\detokenize{utility_functions:utility_functions.f_to_srsf}}\pysiglinewithargsret{\sphinxcode{\sphinxupquote{utility\_functions.}}\sphinxbfcode{\sphinxupquote{f\_to\_srsf}}}{\emph{f}, \emph{time}, \emph{smooth=False}}{}
converts f to a square-root slope function (SRSF)
\begin{quote}\begin{description}
\item[{Parameters}] \leavevmode\begin{itemize}
\item {} 
\sphinxstyleliteralstrong{\sphinxupquote{f}} \textendash{} vector of size N samples

\item {} 
\sphinxstyleliteralstrong{\sphinxupquote{time}} \textendash{} vector of size N describing the sample points

\end{itemize}

\item[{Return type}] \leavevmode
vector

\item[{Return q}] \leavevmode
srsf of f

\end{description}\end{quote}

\end{fulllineitems}

\index{geigen() (in module utility\_functions)@\spxentry{geigen()}\spxextra{in module utility\_functions}}

\begin{fulllineitems}
\phantomsection\label{\detokenize{utility_functions:utility_functions.geigen}}\pysiglinewithargsret{\sphinxcode{\sphinxupquote{utility\_functions.}}\sphinxbfcode{\sphinxupquote{geigen}}}{\emph{Amat}, \emph{Bmat}, \emph{Cmat}}{}
generalized eigenvalue problem of the form

max tr L’AM / sqrt(tr L’BL tr M’CM) w.r.t. L and M

:param Amat numpy ndarray of shape (M,N)
:param Bmat numpy ndarray of shape (M,N)
:param Bmat numpy ndarray of shape (M,N)
\begin{quote}\begin{description}
\item[{Return type}] \leavevmode
numpy ndarray

\item[{Return values}] \leavevmode
eigenvalues

\item[{Return Lmat}] \leavevmode
left eigenvectors

\item[{Return Mmat}] \leavevmode
right eigenvectors

\end{description}\end{quote}

\end{fulllineitems}

\index{gradient\_spline() (in module utility\_functions)@\spxentry{gradient\_spline()}\spxextra{in module utility\_functions}}

\begin{fulllineitems}
\phantomsection\label{\detokenize{utility_functions:utility_functions.gradient_spline}}\pysiglinewithargsret{\sphinxcode{\sphinxupquote{utility\_functions.}}\sphinxbfcode{\sphinxupquote{gradient\_spline}}}{\emph{time}, \emph{f}, \emph{smooth=False}}{}
This function takes the gradient of f using b-spline smoothing
\begin{quote}\begin{description}
\item[{Parameters}] \leavevmode\begin{itemize}
\item {} 
\sphinxstyleliteralstrong{\sphinxupquote{time}} \textendash{} vector of size N describing the sample points

\item {} 
\sphinxstyleliteralstrong{\sphinxupquote{f}} \textendash{} numpy ndarray of shape (M,N) of M functions with N samples

\item {} 
\sphinxstyleliteralstrong{\sphinxupquote{smooth}} \textendash{} smooth data (default = F)

\end{itemize}

\item[{Return type}] \leavevmode
tuple of numpy ndarray

\item[{Return f0}] \leavevmode
smoothed functions functions

\item[{Return g}] \leavevmode
first derivative of each function

\item[{Return g2}] \leavevmode
second derivative of each function

\end{description}\end{quote}

\end{fulllineitems}

\index{innerprod\_q() (in module utility\_functions)@\spxentry{innerprod\_q()}\spxextra{in module utility\_functions}}

\begin{fulllineitems}
\phantomsection\label{\detokenize{utility_functions:utility_functions.innerprod_q}}\pysiglinewithargsret{\sphinxcode{\sphinxupquote{utility\_functions.}}\sphinxbfcode{\sphinxupquote{innerprod\_q}}}{\emph{time}, \emph{q1}, \emph{q2}}{}
calculates the innerproduct between two srsfs

:param time vector descrbing time samples
:param q1 vector of srsf 1
:param q2 vector of srsf 2
\begin{quote}\begin{description}
\item[{Return type}] \leavevmode
scalar

\item[{Return val}] \leavevmode
inner product value

\end{description}\end{quote}

\end{fulllineitems}

\index{invertGamma() (in module utility\_functions)@\spxentry{invertGamma()}\spxextra{in module utility\_functions}}

\begin{fulllineitems}
\phantomsection\label{\detokenize{utility_functions:utility_functions.invertGamma}}\pysiglinewithargsret{\sphinxcode{\sphinxupquote{utility\_functions.}}\sphinxbfcode{\sphinxupquote{invertGamma}}}{\emph{gam}}{}
finds the inverse of the diffeomorphism gamma
\begin{quote}\begin{description}
\item[{Parameters}] \leavevmode
\sphinxstyleliteralstrong{\sphinxupquote{gam}} \textendash{} vector describing the warping function

\item[{Return type}] \leavevmode
vector

\item[{Return gamI}] \leavevmode
inverse of gam

\end{description}\end{quote}

\end{fulllineitems}

\index{optimum\_reparam() (in module utility\_functions)@\spxentry{optimum\_reparam()}\spxextra{in module utility\_functions}}

\begin{fulllineitems}
\phantomsection\label{\detokenize{utility_functions:utility_functions.optimum_reparam}}\pysiglinewithargsret{\sphinxcode{\sphinxupquote{utility\_functions.}}\sphinxbfcode{\sphinxupquote{optimum\_reparam}}}{\emph{q1}, \emph{time}, \emph{q2}, \emph{method='DP'}, \emph{lam=0.0}, \emph{f1o=0.0}, \emph{f2o=0.0}}{}
calculates the warping to align srsf q2 to q1
\begin{quote}\begin{description}
\item[{Parameters}] \leavevmode\begin{itemize}
\item {} 
\sphinxstyleliteralstrong{\sphinxupquote{q1}} \textendash{} vector of size N or array of NxM samples of first SRSF

\item {} 
\sphinxstyleliteralstrong{\sphinxupquote{time}} \textendash{} vector of size N describing the sample points

\item {} 
\sphinxstyleliteralstrong{\sphinxupquote{q2}} \textendash{} vector of size N or array of NxM samples samples of second SRSF

\item {} 
\sphinxstyleliteralstrong{\sphinxupquote{method}} \textendash{} method to apply optimzation (default=”DP”) options are “DP”, “DP2” and “RBFGS”

\item {} 
\sphinxstyleliteralstrong{\sphinxupquote{lam}} \textendash{} controls the amount of elasticity (default = 0.0)

\end{itemize}

\item[{Return type}] \leavevmode
vector

\item[{Return gam}] \leavevmode
describing the warping function used to align q2 with q1

\end{description}\end{quote}

\end{fulllineitems}

\index{optimum\_reparam\_pair() (in module utility\_functions)@\spxentry{optimum\_reparam\_pair()}\spxextra{in module utility\_functions}}

\begin{fulllineitems}
\phantomsection\label{\detokenize{utility_functions:utility_functions.optimum_reparam_pair}}\pysiglinewithargsret{\sphinxcode{\sphinxupquote{utility\_functions.}}\sphinxbfcode{\sphinxupquote{optimum\_reparam\_pair}}}{\emph{q}, \emph{time}, \emph{q1}, \emph{q2}, \emph{lam=0.0}}{}
calculates the warping to align srsf pair q1 and q2 to q
\begin{quote}\begin{description}
\item[{Parameters}] \leavevmode\begin{itemize}
\item {} 
\sphinxstyleliteralstrong{\sphinxupquote{q}} \textendash{} vector of size N or array of NxM samples of first SRSF

\item {} 
\sphinxstyleliteralstrong{\sphinxupquote{time}} \textendash{} vector of size N describing the sample points

\item {} 
\sphinxstyleliteralstrong{\sphinxupquote{q1}} \textendash{} vector of size N or array of NxM samples samples of second SRSF

\item {} 
\sphinxstyleliteralstrong{\sphinxupquote{q2}} \textendash{} vector of size N or array of NxM samples samples of second SRSF

\item {} 
\sphinxstyleliteralstrong{\sphinxupquote{lam}} \textendash{} controls the amount of elasticity (default = 0.0)

\end{itemize}

\item[{Return type}] \leavevmode
vector

\item[{Return gam}] \leavevmode
describing the warping function used to align q2 with q1

\end{description}\end{quote}

\end{fulllineitems}

\index{outlier\_detection() (in module utility\_functions)@\spxentry{outlier\_detection()}\spxextra{in module utility\_functions}}

\begin{fulllineitems}
\phantomsection\label{\detokenize{utility_functions:utility_functions.outlier_detection}}\pysiglinewithargsret{\sphinxcode{\sphinxupquote{utility\_functions.}}\sphinxbfcode{\sphinxupquote{outlier\_detection}}}{\emph{q}, \emph{time}, \emph{mq}, \emph{k=1.5}}{}
calculates outlier’s using geodesic distances of the SRSFs from the median
\begin{quote}\begin{description}
\item[{Parameters}] \leavevmode\begin{itemize}
\item {} 
\sphinxstyleliteralstrong{\sphinxupquote{q}} \textendash{} numpy ndarray of N x M of M SRS functions with N samples

\item {} 
\sphinxstyleliteralstrong{\sphinxupquote{time}} \textendash{} vector of size N describing the sample points

\item {} 
\sphinxstyleliteralstrong{\sphinxupquote{mq}} \textendash{} median calculated using {\hyperref[\detokenize{time_warping:time_warping.srsf_align}]{\sphinxcrossref{\sphinxcode{\sphinxupquote{time\_warping.srsf\_align()}}}}}

\item {} 
\sphinxstyleliteralstrong{\sphinxupquote{k}} \textendash{} cutoff threshold (default = 1.5)

\end{itemize}

\item[{Returns}] \leavevmode
q\_outlier: outlier functions

\end{description}\end{quote}

\end{fulllineitems}

\index{randomGamma() (in module utility\_functions)@\spxentry{randomGamma()}\spxextra{in module utility\_functions}}

\begin{fulllineitems}
\phantomsection\label{\detokenize{utility_functions:utility_functions.randomGamma}}\pysiglinewithargsret{\sphinxcode{\sphinxupquote{utility\_functions.}}\sphinxbfcode{\sphinxupquote{randomGamma}}}{\emph{gam}, \emph{num}}{}
generates random warping functions
\begin{quote}\begin{description}
\item[{Parameters}] \leavevmode\begin{itemize}
\item {} 
\sphinxstyleliteralstrong{\sphinxupquote{gam}} \textendash{} numpy ndarray of N x M of M of warping functions

\item {} 
\sphinxstyleliteralstrong{\sphinxupquote{num}} \textendash{} number of random functions

\end{itemize}

\item[{Returns}] \leavevmode
rgam: random warping functions

\end{description}\end{quote}

\end{fulllineitems}

\index{resamplefunction() (in module utility\_functions)@\spxentry{resamplefunction()}\spxextra{in module utility\_functions}}

\begin{fulllineitems}
\phantomsection\label{\detokenize{utility_functions:utility_functions.resamplefunction}}\pysiglinewithargsret{\sphinxcode{\sphinxupquote{utility\_functions.}}\sphinxbfcode{\sphinxupquote{resamplefunction}}}{\emph{x}, \emph{n}}{}
resample function using n points
\begin{quote}\begin{description}
\item[{Parameters}] \leavevmode\begin{itemize}
\item {} 
\sphinxstyleliteralstrong{\sphinxupquote{x}} \textendash{} functions

\item {} 
\sphinxstyleliteralstrong{\sphinxupquote{n}} \textendash{} number of points

\end{itemize}

\item[{Return type}] \leavevmode
numpy array

\item[{Return xn}] \leavevmode
resampled function

\end{description}\end{quote}

\end{fulllineitems}

\index{rgam() (in module utility\_functions)@\spxentry{rgam()}\spxextra{in module utility\_functions}}

\begin{fulllineitems}
\phantomsection\label{\detokenize{utility_functions:utility_functions.rgam}}\pysiglinewithargsret{\sphinxcode{\sphinxupquote{utility\_functions.}}\sphinxbfcode{\sphinxupquote{rgam}}}{\emph{N}, \emph{sigma}, \emph{num}}{}
Generates random warping functions
\begin{quote}\begin{description}
\item[{Parameters}] \leavevmode\begin{itemize}
\item {} 
\sphinxstyleliteralstrong{\sphinxupquote{N}} \textendash{} length of warping function

\item {} 
\sphinxstyleliteralstrong{\sphinxupquote{sigma}} \textendash{} variance of warping functions

\item {} 
\sphinxstyleliteralstrong{\sphinxupquote{num}} \textendash{} number of warping functions

\end{itemize}

\item[{Returns}] \leavevmode
gam: numpy ndarray of warping functions

\end{description}\end{quote}

\end{fulllineitems}

\index{smooth\_data() (in module utility\_functions)@\spxentry{smooth\_data()}\spxextra{in module utility\_functions}}

\begin{fulllineitems}
\phantomsection\label{\detokenize{utility_functions:utility_functions.smooth_data}}\pysiglinewithargsret{\sphinxcode{\sphinxupquote{utility\_functions.}}\sphinxbfcode{\sphinxupquote{smooth\_data}}}{\emph{f}, \emph{sparam}}{}
This function smooths a collection of functions using a box filter
\begin{quote}\begin{description}
\item[{Parameters}] \leavevmode\begin{itemize}
\item {} 
\sphinxstyleliteralstrong{\sphinxupquote{f}} \textendash{} numpy ndarray of shape (M,N) of M functions with N samples

\item {} 
\sphinxstyleliteralstrong{\sphinxupquote{sparam}} \textendash{} Number of times to run box filter (default = 25)

\end{itemize}

\item[{Return type}] \leavevmode
numpy ndarray

\item[{Return f}] \leavevmode
smoothed functions functions

\end{description}\end{quote}

\end{fulllineitems}

\index{srsf\_to\_f() (in module utility\_functions)@\spxentry{srsf\_to\_f()}\spxextra{in module utility\_functions}}

\begin{fulllineitems}
\phantomsection\label{\detokenize{utility_functions:utility_functions.srsf_to_f}}\pysiglinewithargsret{\sphinxcode{\sphinxupquote{utility\_functions.}}\sphinxbfcode{\sphinxupquote{srsf\_to\_f}}}{\emph{q}, \emph{time}, \emph{f0=0.0}}{}
converts q (srsf) to a function
\begin{quote}\begin{description}
\item[{Parameters}] \leavevmode\begin{itemize}
\item {} 
\sphinxstyleliteralstrong{\sphinxupquote{q}} \textendash{} vector of size N samples of srsf

\item {} 
\sphinxstyleliteralstrong{\sphinxupquote{time}} \textendash{} vector of size N describing time sample points

\item {} 
\sphinxstyleliteralstrong{\sphinxupquote{f0}} \textendash{} initial value

\end{itemize}

\item[{Return type}] \leavevmode
vector

\item[{Return f}] \leavevmode
function

\end{description}\end{quote}

\end{fulllineitems}

\index{update\_progress() (in module utility\_functions)@\spxentry{update\_progress()}\spxextra{in module utility\_functions}}

\begin{fulllineitems}
\phantomsection\label{\detokenize{utility_functions:utility_functions.update_progress}}\pysiglinewithargsret{\sphinxcode{\sphinxupquote{utility\_functions.}}\sphinxbfcode{\sphinxupquote{update\_progress}}}{\emph{progress}}{}
This function creates a progress bar
\begin{quote}\begin{description}
\item[{Parameters}] \leavevmode
\sphinxstyleliteralstrong{\sphinxupquote{progress}} \textendash{} fraction of progress

\end{description}\end{quote}

\end{fulllineitems}

\index{warp\_f\_gamma() (in module utility\_functions)@\spxentry{warp\_f\_gamma()}\spxextra{in module utility\_functions}}

\begin{fulllineitems}
\phantomsection\label{\detokenize{utility_functions:utility_functions.warp_f_gamma}}\pysiglinewithargsret{\sphinxcode{\sphinxupquote{utility\_functions.}}\sphinxbfcode{\sphinxupquote{warp\_f\_gamma}}}{\emph{time}, \emph{f}, \emph{gam}}{}
warps a function f by gam

:param time vector describing time samples
:param q vector describing srsf
:param gam vector describing warping function
\begin{quote}\begin{description}
\item[{Return type}] \leavevmode
numpy ndarray

\item[{Return f\_temp}] \leavevmode
warped srsf

\end{description}\end{quote}

\end{fulllineitems}

\index{warp\_q\_gamma() (in module utility\_functions)@\spxentry{warp\_q\_gamma()}\spxextra{in module utility\_functions}}

\begin{fulllineitems}
\phantomsection\label{\detokenize{utility_functions:utility_functions.warp_q_gamma}}\pysiglinewithargsret{\sphinxcode{\sphinxupquote{utility\_functions.}}\sphinxbfcode{\sphinxupquote{warp\_q\_gamma}}}{\emph{time}, \emph{q}, \emph{gam}}{}
warps a srsf q by gam

:param time vector describing time samples
:param q vector describing srsf
:param gam vector describing warping function
\begin{quote}\begin{description}
\item[{Return type}] \leavevmode
numpy ndarray

\item[{Return q\_temp}] \leavevmode
warped srsf

\end{description}\end{quote}

\end{fulllineitems}

\index{zero\_crossing() (in module utility\_functions)@\spxentry{zero\_crossing()}\spxextra{in module utility\_functions}}

\begin{fulllineitems}
\phantomsection\label{\detokenize{utility_functions:utility_functions.zero_crossing}}\pysiglinewithargsret{\sphinxcode{\sphinxupquote{utility\_functions.}}\sphinxbfcode{\sphinxupquote{zero\_crossing}}}{\emph{Y}, \emph{q}, \emph{bt}, \emph{time}, \emph{y\_max}, \emph{y\_min}, \emph{gmax}, \emph{gmin}}{}
finds zero-crossing of optimal gamma, gam = s*gmax + (1-s)*gmin
from elastic regression model
\begin{quote}\begin{description}
\item[{Parameters}] \leavevmode\begin{itemize}
\item {} 
\sphinxstyleliteralstrong{\sphinxupquote{Y}} \textendash{} response

\item {} 
\sphinxstyleliteralstrong{\sphinxupquote{q}} \textendash{} predicitve function

\item {} 
\sphinxstyleliteralstrong{\sphinxupquote{bt}} \textendash{} basis function

\item {} 
\sphinxstyleliteralstrong{\sphinxupquote{time}} \textendash{} time samples

\item {} 
\sphinxstyleliteralstrong{\sphinxupquote{y\_max}} \textendash{} maximum repsonse for warping function gmax

\item {} 
\sphinxstyleliteralstrong{\sphinxupquote{y\_min}} \textendash{} minimum response for warping function gmin

\item {} 
\sphinxstyleliteralstrong{\sphinxupquote{gmax}} \textendash{} max warping function

\item {} 
\sphinxstyleliteralstrong{\sphinxupquote{gmin}} \textendash{} min warping fucntion

\end{itemize}

\item[{Return type}] \leavevmode
numpy array

\item[{Return gamma}] \leavevmode
optimal warping function

\end{description}\end{quote}

\end{fulllineitems}



\chapter{Curve Functions}
\label{\detokenize{curve_functions:module-curve_functions}}\label{\detokenize{curve_functions:curve-functions}}\label{\detokenize{curve_functions::doc}}\index{curve\_functions (module)@\spxentry{curve\_functions}\spxextra{module}}
functions for SRVF curve manipulations

moduleauthor:: Derek Tucker \textless{}\sphinxhref{mailto:jdtuck@sandia.gov}{jdtuck@sandia.gov}\textgreater{}
\index{calc\_j() (in module curve\_functions)@\spxentry{calc\_j()}\spxextra{in module curve\_functions}}

\begin{fulllineitems}
\phantomsection\label{\detokenize{curve_functions:curve_functions.calc_j}}\pysiglinewithargsret{\sphinxcode{\sphinxupquote{curve\_functions.}}\sphinxbfcode{\sphinxupquote{calc\_j}}}{\emph{basis}}{}
Calculates Jacobian matrix from normal basis
\begin{quote}\begin{description}
\item[{Parameters}] \leavevmode
\sphinxstyleliteralstrong{\sphinxupquote{basis}} \textendash{} list of numpy ndarray of shape (2,M) of M samples basis

\item[{Return type}] \leavevmode
numpy ndarray

\item[{Return j}] \leavevmode
Jacobian

\end{description}\end{quote}

\end{fulllineitems}

\index{calculate\_variance() (in module curve\_functions)@\spxentry{calculate\_variance()}\spxextra{in module curve\_functions}}

\begin{fulllineitems}
\phantomsection\label{\detokenize{curve_functions:curve_functions.calculate_variance}}\pysiglinewithargsret{\sphinxcode{\sphinxupquote{curve\_functions.}}\sphinxbfcode{\sphinxupquote{calculate\_variance}}}{\emph{beta}}{}
This function calculates variance of curve beta
\begin{quote}\begin{description}
\item[{Parameters}] \leavevmode
\sphinxstyleliteralstrong{\sphinxupquote{beta}} \textendash{} numpy ndarray of shape (2,M) of M samples

\item[{Return type}] \leavevmode
numpy ndarray

\item[{Return variance}] \leavevmode
variance

\end{description}\end{quote}

\end{fulllineitems}

\index{calculatecentroid() (in module curve\_functions)@\spxentry{calculatecentroid()}\spxextra{in module curve\_functions}}

\begin{fulllineitems}
\phantomsection\label{\detokenize{curve_functions:curve_functions.calculatecentroid}}\pysiglinewithargsret{\sphinxcode{\sphinxupquote{curve\_functions.}}\sphinxbfcode{\sphinxupquote{calculatecentroid}}}{\emph{beta}}{}
This function calculates centroid of a parameterized curve
\begin{quote}\begin{description}
\item[{Parameters}] \leavevmode
\sphinxstyleliteralstrong{\sphinxupquote{beta}} \textendash{} numpy ndarray of shape (2,M) of M samples

\item[{Return type}] \leavevmode
numpy ndarray

\item[{Return centroid}] \leavevmode
center coordinates

\end{description}\end{quote}

\end{fulllineitems}

\index{curve\_to\_q() (in module curve\_functions)@\spxentry{curve\_to\_q()}\spxextra{in module curve\_functions}}

\begin{fulllineitems}
\phantomsection\label{\detokenize{curve_functions:curve_functions.curve_to_q}}\pysiglinewithargsret{\sphinxcode{\sphinxupquote{curve\_functions.}}\sphinxbfcode{\sphinxupquote{curve\_to\_q}}}{\emph{beta}}{}
This function converts curve beta to srvf q
\begin{quote}\begin{description}
\item[{Parameters}] \leavevmode
\sphinxstyleliteralstrong{\sphinxupquote{beta}} \textendash{} numpy ndarray of shape (2,M) of M samples

\item[{Return type}] \leavevmode
numpy ndarray

\item[{Return q}] \leavevmode
srvf of curve

\end{description}\end{quote}

\end{fulllineitems}

\index{curve\_zero\_crossing() (in module curve\_functions)@\spxentry{curve\_zero\_crossing()}\spxextra{in module curve\_functions}}

\begin{fulllineitems}
\phantomsection\label{\detokenize{curve_functions:curve_functions.curve_zero_crossing}}\pysiglinewithargsret{\sphinxcode{\sphinxupquote{curve\_functions.}}\sphinxbfcode{\sphinxupquote{curve\_zero\_crossing}}}{\emph{Y}, \emph{beta}, \emph{bt}, \emph{y\_max}, \emph{y\_min}, \emph{gmax}, \emph{gmin}}{}
finds zero-crossing of optimal gamma, gam = s*gmax + (1-s)*gmin
from elastic curve regression model
\begin{quote}\begin{description}
\item[{Parameters}] \leavevmode\begin{itemize}
\item {} 
\sphinxstyleliteralstrong{\sphinxupquote{Y}} \textendash{} response

\item {} 
\sphinxstyleliteralstrong{\sphinxupquote{beta}} \textendash{} predicitve function

\item {} 
\sphinxstyleliteralstrong{\sphinxupquote{bt}} \textendash{} basis function

\item {} 
\sphinxstyleliteralstrong{\sphinxupquote{y\_max}} \textendash{} maximum repsonse for warping function gmax

\item {} 
\sphinxstyleliteralstrong{\sphinxupquote{y\_min}} \textendash{} minimum response for warping function gmin

\item {} 
\sphinxstyleliteralstrong{\sphinxupquote{gmax}} \textendash{} max warping function

\item {} 
\sphinxstyleliteralstrong{\sphinxupquote{gmin}} \textendash{} min warping fucntion

\end{itemize}

\item[{Return type}] \leavevmode
numpy array

\item[{Return gamma}] \leavevmode
optimal warping function

\item[{Return O\_hat}] \leavevmode
rotation matrix

\end{description}\end{quote}

\end{fulllineitems}

\index{find\_basis\_normal() (in module curve\_functions)@\spxentry{find\_basis\_normal()}\spxextra{in module curve\_functions}}

\begin{fulllineitems}
\phantomsection\label{\detokenize{curve_functions:curve_functions.find_basis_normal}}\pysiglinewithargsret{\sphinxcode{\sphinxupquote{curve\_functions.}}\sphinxbfcode{\sphinxupquote{find\_basis\_normal}}}{\emph{q}}{}
Finds the basis normal to the srvf
\begin{quote}\begin{description}
\item[{Parameters}] \leavevmode
\sphinxstyleliteralstrong{\sphinxupquote{q1}} \textendash{} numpy ndarray of shape (2,M) of M samples

\item[{Return type}] \leavevmode
list of numpy ndarray

\item[{Return basis}] \leavevmode
list containing basis vectors

\end{description}\end{quote}

\end{fulllineitems}

\index{find\_best\_rotation() (in module curve\_functions)@\spxentry{find\_best\_rotation()}\spxextra{in module curve\_functions}}

\begin{fulllineitems}
\phantomsection\label{\detokenize{curve_functions:curve_functions.find_best_rotation}}\pysiglinewithargsret{\sphinxcode{\sphinxupquote{curve\_functions.}}\sphinxbfcode{\sphinxupquote{find\_best\_rotation}}}{\emph{q1}, \emph{q2}}{}
This function calculates the best rotation between two srvfs using
procustes rigid alignment
\begin{quote}\begin{description}
\item[{Parameters}] \leavevmode\begin{itemize}
\item {} 
\sphinxstyleliteralstrong{\sphinxupquote{q1}} \textendash{} numpy ndarray of shape (2,M) of M samples

\item {} 
\sphinxstyleliteralstrong{\sphinxupquote{q2}} \textendash{} numpy ndarray of shape (2,M) of M samples

\end{itemize}

\item[{Return type}] \leavevmode
numpy ndarray

\item[{Return q2new}] \leavevmode
optimal rotated q2 to q1

\item[{Return R}] \leavevmode
rotation matrix

\end{description}\end{quote}

\end{fulllineitems}

\index{find\_rotation\_and\_seed\_coord() (in module curve\_functions)@\spxentry{find\_rotation\_and\_seed\_coord()}\spxextra{in module curve\_functions}}

\begin{fulllineitems}
\phantomsection\label{\detokenize{curve_functions:curve_functions.find_rotation_and_seed_coord}}\pysiglinewithargsret{\sphinxcode{\sphinxupquote{curve\_functions.}}\sphinxbfcode{\sphinxupquote{find\_rotation\_and\_seed\_coord}}}{\emph{beta1}, \emph{beta2}}{}
This function returns a candidate list of optimally oriented and
registered (seed) shapes w.r.t. beta1
\begin{quote}\begin{description}
\item[{Parameters}] \leavevmode\begin{itemize}
\item {} 
\sphinxstyleliteralstrong{\sphinxupquote{beta1}} \textendash{} numpy ndarray of shape (2,M) of M samples

\item {} 
\sphinxstyleliteralstrong{\sphinxupquote{beta2}} \textendash{} numpy ndarray of shape (2,M) of M samples

\end{itemize}

\item[{Return type}] \leavevmode
numpy ndarray

\item[{Return beta2new}] \leavevmode
optimal rotated beta2 to beta1

\item[{Return O}] \leavevmode
rotation matrix

\item[{Return tau}] \leavevmode
seed

\end{description}\end{quote}

\end{fulllineitems}

\index{find\_rotation\_and\_seed\_q() (in module curve\_functions)@\spxentry{find\_rotation\_and\_seed\_q()}\spxextra{in module curve\_functions}}

\begin{fulllineitems}
\phantomsection\label{\detokenize{curve_functions:curve_functions.find_rotation_and_seed_q}}\pysiglinewithargsret{\sphinxcode{\sphinxupquote{curve\_functions.}}\sphinxbfcode{\sphinxupquote{find\_rotation\_and\_seed\_q}}}{\emph{q1}, \emph{q2}}{}
This function returns a candidate list of optimally oriented and
registered (seed) shapes w.r.t. beta1
\begin{quote}\begin{description}
\item[{Parameters}] \leavevmode\begin{itemize}
\item {} 
\sphinxstyleliteralstrong{\sphinxupquote{q1}} \textendash{} numpy ndarray of shape (2,M) of M samples

\item {} 
\sphinxstyleliteralstrong{\sphinxupquote{q2}} \textendash{} numpy ndarray of shape (2,M) of M samples

\end{itemize}

\item[{Return type}] \leavevmode
numpy ndarray

\item[{Return beta2new}] \leavevmode
optimal rotated beta2 to beta1

\item[{Return O}] \leavevmode
rotation matrix

\item[{Return tau}] \leavevmode
seed

\end{description}\end{quote}

\end{fulllineitems}

\index{gram\_schmidt() (in module curve\_functions)@\spxentry{gram\_schmidt()}\spxextra{in module curve\_functions}}

\begin{fulllineitems}
\phantomsection\label{\detokenize{curve_functions:curve_functions.gram_schmidt}}\pysiglinewithargsret{\sphinxcode{\sphinxupquote{curve\_functions.}}\sphinxbfcode{\sphinxupquote{gram\_schmidt}}}{\emph{basis}}{}
Performs Gram Schmidt Orthogonlization of a basis\_o
\begin{quote}
\begin{quote}\begin{description}
\item[{param basis}] \leavevmode
list of numpy ndarray of shape (2,M) of M samples

\item[{rtype}] \leavevmode
list of numpy ndarray

\item[{return basis\_o}] \leavevmode
orthogonlized basis

\end{description}\end{quote}
\end{quote}

\end{fulllineitems}

\index{group\_action\_by\_gamma() (in module curve\_functions)@\spxentry{group\_action\_by\_gamma()}\spxextra{in module curve\_functions}}

\begin{fulllineitems}
\phantomsection\label{\detokenize{curve_functions:curve_functions.group_action_by_gamma}}\pysiglinewithargsret{\sphinxcode{\sphinxupquote{curve\_functions.}}\sphinxbfcode{\sphinxupquote{group\_action\_by\_gamma}}}{\emph{q}, \emph{gamma}}{}
This function reparamerized srvf q by gamma
\begin{quote}\begin{description}
\item[{Parameters}] \leavevmode\begin{itemize}
\item {} 
\sphinxstyleliteralstrong{\sphinxupquote{f}} \textendash{} numpy ndarray of shape (2,M) of M samples

\item {} 
\sphinxstyleliteralstrong{\sphinxupquote{gamma}} \textendash{} numpy ndarray of shape (2,M) of M samples

\end{itemize}

\item[{Return type}] \leavevmode
numpy ndarray

\item[{Return qn}] \leavevmode
reparatermized srvf

\end{description}\end{quote}

\end{fulllineitems}

\index{group\_action\_by\_gamma\_coord() (in module curve\_functions)@\spxentry{group\_action\_by\_gamma\_coord()}\spxextra{in module curve\_functions}}

\begin{fulllineitems}
\phantomsection\label{\detokenize{curve_functions:curve_functions.group_action_by_gamma_coord}}\pysiglinewithargsret{\sphinxcode{\sphinxupquote{curve\_functions.}}\sphinxbfcode{\sphinxupquote{group\_action\_by\_gamma\_coord}}}{\emph{f}, \emph{gamma}}{}
This function reparamerized curve f by gamma
\begin{quote}\begin{description}
\item[{Parameters}] \leavevmode\begin{itemize}
\item {} 
\sphinxstyleliteralstrong{\sphinxupquote{f}} \textendash{} numpy ndarray of shape (2,M) of M samples

\item {} 
\sphinxstyleliteralstrong{\sphinxupquote{gamma}} \textendash{} numpy ndarray of shape (2,M) of M samples

\end{itemize}

\item[{Return type}] \leavevmode
numpy ndarray

\item[{Return fn}] \leavevmode
reparatermized curve

\end{description}\end{quote}

\end{fulllineitems}

\index{innerprod\_q2() (in module curve\_functions)@\spxentry{innerprod\_q2()}\spxextra{in module curve\_functions}}

\begin{fulllineitems}
\phantomsection\label{\detokenize{curve_functions:curve_functions.innerprod_q2}}\pysiglinewithargsret{\sphinxcode{\sphinxupquote{curve\_functions.}}\sphinxbfcode{\sphinxupquote{innerprod\_q2}}}{\emph{q1}, \emph{q2}}{}
This function calculates the inner product in srvf space
\begin{quote}\begin{description}
\item[{Parameters}] \leavevmode\begin{itemize}
\item {} 
\sphinxstyleliteralstrong{\sphinxupquote{q1}} \textendash{} numpy ndarray of shape (2,M) of M samples

\item {} 
\sphinxstyleliteralstrong{\sphinxupquote{q2}} \textendash{} numpy ndarray of shape (2,M) of M samples

\end{itemize}

\item[{Return type}] \leavevmode
numpy ndarray

\item[{Return val}] \leavevmode
inner product

\end{description}\end{quote}

\end{fulllineitems}

\index{inverse\_exp() (in module curve\_functions)@\spxentry{inverse\_exp()}\spxextra{in module curve\_functions}}

\begin{fulllineitems}
\phantomsection\label{\detokenize{curve_functions:curve_functions.inverse_exp}}\pysiglinewithargsret{\sphinxcode{\sphinxupquote{curve\_functions.}}\sphinxbfcode{\sphinxupquote{inverse\_exp}}}{\emph{q1}, \emph{q2}, \emph{beta2}}{}
Calculate the inverse exponential to obtain a shooting vector from
q1 to q2 in shape space of open curves
\begin{quote}\begin{description}
\item[{Parameters}] \leavevmode\begin{itemize}
\item {} 
\sphinxstyleliteralstrong{\sphinxupquote{q1}} \textendash{} numpy ndarray of shape (2,M) of M samples

\item {} 
\sphinxstyleliteralstrong{\sphinxupquote{q2}} \textendash{} numpy ndarray of shape (2,M) of M samples

\item {} 
\sphinxstyleliteralstrong{\sphinxupquote{beta2}} \textendash{} numpy ndarray of shape (2,M) of M samples

\end{itemize}

\item[{Return type}] \leavevmode
numpy ndarray

\item[{Return v}] \leavevmode
shooting vectors

\end{description}\end{quote}

\end{fulllineitems}

\index{inverse\_exp\_coord() (in module curve\_functions)@\spxentry{inverse\_exp\_coord()}\spxextra{in module curve\_functions}}

\begin{fulllineitems}
\phantomsection\label{\detokenize{curve_functions:curve_functions.inverse_exp_coord}}\pysiglinewithargsret{\sphinxcode{\sphinxupquote{curve\_functions.}}\sphinxbfcode{\sphinxupquote{inverse\_exp\_coord}}}{\emph{beta1}, \emph{beta2}}{}
Calculate the inverse exponential to obtain a shooting vector from
beta1 to beta2 in shape space of open curves
\begin{quote}\begin{description}
\item[{Parameters}] \leavevmode\begin{itemize}
\item {} 
\sphinxstyleliteralstrong{\sphinxupquote{beta1}} \textendash{} numpy ndarray of shape (2,M) of M samples

\item {} 
\sphinxstyleliteralstrong{\sphinxupquote{beta2}} \textendash{} numpy ndarray of shape (2,M) of M samples

\end{itemize}

\item[{Return type}] \leavevmode
numpy ndarray

\item[{Return v}] \leavevmode
shooting vectors

\item[{Return dist}] \leavevmode
distance

\end{description}\end{quote}

\end{fulllineitems}

\index{optimum\_reparam\_curve() (in module curve\_functions)@\spxentry{optimum\_reparam\_curve()}\spxextra{in module curve\_functions}}

\begin{fulllineitems}
\phantomsection\label{\detokenize{curve_functions:curve_functions.optimum_reparam_curve}}\pysiglinewithargsret{\sphinxcode{\sphinxupquote{curve\_functions.}}\sphinxbfcode{\sphinxupquote{optimum\_reparam\_curve}}}{\emph{q1}, \emph{q2}, \emph{lam=0.0}}{}
calculates the warping to align srsf q2 to q1
\begin{quote}\begin{description}
\item[{Parameters}] \leavevmode\begin{itemize}
\item {} 
\sphinxstyleliteralstrong{\sphinxupquote{q1}} \textendash{} matrix of size nxN or array of NxM samples of first SRVF

\item {} 
\sphinxstyleliteralstrong{\sphinxupquote{time}} \textendash{} vector of size N describing the sample points

\item {} 
\sphinxstyleliteralstrong{\sphinxupquote{q2}} \textendash{} matrix of size nxN or array of NxM samples samples of second SRVF

\item {} 
\sphinxstyleliteralstrong{\sphinxupquote{lam}} \textendash{} controls the amount of elasticity (default = 0.0)

\end{itemize}

\item[{Return type}] \leavevmode
vector

\item[{Return gam}] \leavevmode
describing the warping function used to align q2 with q1

\end{description}\end{quote}

\end{fulllineitems}

\index{parallel\_translate() (in module curve\_functions)@\spxentry{parallel\_translate()}\spxextra{in module curve\_functions}}

\begin{fulllineitems}
\phantomsection\label{\detokenize{curve_functions:curve_functions.parallel_translate}}\pysiglinewithargsret{\sphinxcode{\sphinxupquote{curve\_functions.}}\sphinxbfcode{\sphinxupquote{parallel\_translate}}}{\emph{w}, \emph{q1}, \emph{q2}, \emph{basis}, \emph{mode=0}}{}
parallel translates q1 and q2 along manifold
\begin{quote}\begin{description}
\item[{Parameters}] \leavevmode\begin{itemize}
\item {} 
\sphinxstyleliteralstrong{\sphinxupquote{w}} \textendash{} numpy ndarray of shape (2,M) of M samples

\item {} 
\sphinxstyleliteralstrong{\sphinxupquote{q1}} \textendash{} numpy ndarray of shape (2,M) of M samples

\item {} 
\sphinxstyleliteralstrong{\sphinxupquote{q2}} \textendash{} numpy ndarray of shape (2,M) of M samples

\item {} 
\sphinxstyleliteralstrong{\sphinxupquote{basis}} \textendash{} list of numpy ndarray of shape (2,M) of M samples

\item {} 
\sphinxstyleliteralstrong{\sphinxupquote{mode}} \textendash{} open 0 or closed curves 1 (default 0)

\end{itemize}

\item[{Return type}] \leavevmode
numpy ndarray

\item[{Return wbar}] \leavevmode
translated vector

\end{description}\end{quote}

\end{fulllineitems}

\index{pre\_proc\_curve() (in module curve\_functions)@\spxentry{pre\_proc\_curve()}\spxextra{in module curve\_functions}}

\begin{fulllineitems}
\phantomsection\label{\detokenize{curve_functions:curve_functions.pre_proc_curve}}\pysiglinewithargsret{\sphinxcode{\sphinxupquote{curve\_functions.}}\sphinxbfcode{\sphinxupquote{pre\_proc\_curve}}}{\emph{beta}, \emph{T=100}}{}
This function prepcoessed a curve beta to set of closed curves
\begin{quote}\begin{description}
\item[{Parameters}] \leavevmode\begin{itemize}
\item {} 
\sphinxstyleliteralstrong{\sphinxupquote{beta}} \textendash{} numpy ndarray of shape (2,M) of M samples

\item {} 
\sphinxstyleliteralstrong{\sphinxupquote{T}} \textendash{} number of samples (default = 100)

\end{itemize}

\item[{Return type}] \leavevmode
numpy ndarray

\item[{Return betanew}] \leavevmode
projected beta

\item[{Return qnew}] \leavevmode
projected srvf

\item[{Return A}] \leavevmode
alignment matrix (not used currently)

\end{description}\end{quote}

\end{fulllineitems}

\index{project\_curve() (in module curve\_functions)@\spxentry{project\_curve()}\spxextra{in module curve\_functions}}

\begin{fulllineitems}
\phantomsection\label{\detokenize{curve_functions:curve_functions.project_curve}}\pysiglinewithargsret{\sphinxcode{\sphinxupquote{curve\_functions.}}\sphinxbfcode{\sphinxupquote{project\_curve}}}{\emph{q}}{}
This function projects srvf q to set of close curves
\begin{quote}\begin{description}
\item[{Parameters}] \leavevmode
\sphinxstyleliteralstrong{\sphinxupquote{q}} \textendash{} numpy ndarray of shape (2,M) of M samples

\item[{Return type}] \leavevmode
numpy ndarray

\item[{Return qproj}] \leavevmode
project srvf

\end{description}\end{quote}

\end{fulllineitems}

\index{project\_tangent() (in module curve\_functions)@\spxentry{project\_tangent()}\spxextra{in module curve\_functions}}

\begin{fulllineitems}
\phantomsection\label{\detokenize{curve_functions:curve_functions.project_tangent}}\pysiglinewithargsret{\sphinxcode{\sphinxupquote{curve\_functions.}}\sphinxbfcode{\sphinxupquote{project\_tangent}}}{\emph{w}, \emph{q}, \emph{basis}}{}
projects srvf to tangent space w using basis
\begin{quote}\begin{description}
\item[{Parameters}] \leavevmode\begin{itemize}
\item {} 
\sphinxstyleliteralstrong{\sphinxupquote{w}} \textendash{} numpy ndarray of shape (2,M) of M samples

\item {} 
\sphinxstyleliteralstrong{\sphinxupquote{q}} \textendash{} numpy ndarray of shape (2,M) of M samples

\item {} 
\sphinxstyleliteralstrong{\sphinxupquote{basis}} \textendash{} list of numpy ndarray of shape (2,M) of M samples

\end{itemize}

\item[{Return type}] \leavevmode
numpy ndarray

\item[{Return wproj}] \leavevmode
projected q

\end{description}\end{quote}

\end{fulllineitems}

\index{psi() (in module curve\_functions)@\spxentry{psi()}\spxextra{in module curve\_functions}}

\begin{fulllineitems}
\phantomsection\label{\detokenize{curve_functions:curve_functions.psi}}\pysiglinewithargsret{\sphinxcode{\sphinxupquote{curve\_functions.}}\sphinxbfcode{\sphinxupquote{psi}}}{\emph{x}, \emph{a}, \emph{q}}{}
This function formats variance output
\begin{quote}\begin{description}
\item[{Parameters}] \leavevmode\begin{itemize}
\item {} 
\sphinxstyleliteralstrong{\sphinxupquote{x}} \textendash{} numpy ndarray of shape (2,M) of M samples curve

\item {} 
\sphinxstyleliteralstrong{\sphinxupquote{a}} \textendash{} numpy ndarray of shape (2,1) mean

\item {} 
\sphinxstyleliteralstrong{\sphinxupquote{q}} \textendash{} numpy ndarray of shape (2,M) of M samples srvf

\end{itemize}

\item[{Return type}] \leavevmode
numpy ndarray

\item[{Return psi1}] \leavevmode
variance

\item[{Return psi2}] \leavevmode
cross variance

\item[{Return psi3}] \leavevmode
curve end

\item[{Return psi4}] \leavevmode
curve end

\end{description}\end{quote}

\end{fulllineitems}

\index{q\_to\_curve() (in module curve\_functions)@\spxentry{q\_to\_curve()}\spxextra{in module curve\_functions}}

\begin{fulllineitems}
\phantomsection\label{\detokenize{curve_functions:curve_functions.q_to_curve}}\pysiglinewithargsret{\sphinxcode{\sphinxupquote{curve\_functions.}}\sphinxbfcode{\sphinxupquote{q\_to\_curve}}}{\emph{q}}{}
This function converts srvf to beta
\begin{quote}\begin{description}
\item[{Parameters}] \leavevmode
\sphinxstyleliteralstrong{\sphinxupquote{q}} \textendash{} numpy ndarray of shape (2,M) of M samples

\item[{Return type}] \leavevmode
numpy ndarray

\item[{Return beta}] \leavevmode
parameterized curve

\end{description}\end{quote}

\end{fulllineitems}

\index{resamplecurve() (in module curve\_functions)@\spxentry{resamplecurve()}\spxextra{in module curve\_functions}}

\begin{fulllineitems}
\phantomsection\label{\detokenize{curve_functions:curve_functions.resamplecurve}}\pysiglinewithargsret{\sphinxcode{\sphinxupquote{curve\_functions.}}\sphinxbfcode{\sphinxupquote{resamplecurve}}}{\emph{x}, \emph{N=100}}{}
This function resamples a curve to have N samples
\begin{quote}\begin{description}
\item[{Parameters}] \leavevmode\begin{itemize}
\item {} 
\sphinxstyleliteralstrong{\sphinxupquote{x}} \textendash{} numpy ndarray of shape (2,M) of M samples

\item {} 
\sphinxstyleliteralstrong{\sphinxupquote{N}} \textendash{} Number of samples for new curve (default = 100)

\end{itemize}

\item[{Return type}] \leavevmode
numpy ndarray

\item[{Return xn}] \leavevmode
resampled curve

\end{description}\end{quote}

\end{fulllineitems}

\index{scale\_curve() (in module curve\_functions)@\spxentry{scale\_curve()}\spxextra{in module curve\_functions}}

\begin{fulllineitems}
\phantomsection\label{\detokenize{curve_functions:curve_functions.scale_curve}}\pysiglinewithargsret{\sphinxcode{\sphinxupquote{curve\_functions.}}\sphinxbfcode{\sphinxupquote{scale\_curve}}}{\emph{beta}}{}
scales curve to length 1
\begin{quote}\begin{description}
\item[{Parameters}] \leavevmode
\sphinxstyleliteralstrong{\sphinxupquote{beta}} \textendash{} numpy ndarray of shape (2,M) of M samples

\item[{Return type}] \leavevmode
numpy ndarray

\item[{Return beta\_scaled}] \leavevmode
scaled curve

\item[{Return scale}] \leavevmode
scale factor used

\end{description}\end{quote}

\end{fulllineitems}

\index{shift\_f() (in module curve\_functions)@\spxentry{shift\_f()}\spxextra{in module curve\_functions}}

\begin{fulllineitems}
\phantomsection\label{\detokenize{curve_functions:curve_functions.shift_f}}\pysiglinewithargsret{\sphinxcode{\sphinxupquote{curve\_functions.}}\sphinxbfcode{\sphinxupquote{shift\_f}}}{\emph{f}, \emph{tau}}{}
shifts a curve f by tau
\begin{quote}\begin{description}
\item[{Parameters}] \leavevmode\begin{itemize}
\item {} 
\sphinxstyleliteralstrong{\sphinxupquote{f}} \textendash{} numpy ndarray of shape (2,M) of M samples

\item {} 
\sphinxstyleliteralstrong{\sphinxupquote{tau}} \textendash{} scalar

\end{itemize}

\item[{Return type}] \leavevmode
numpy ndarray

\item[{Return fn}] \leavevmode
shifted curve

\end{description}\end{quote}

\end{fulllineitems}


References:
\begin{quote}

Tucker, J. D. 2014, Functional Component Analysis and Regression using Elastic
Methods. Ph.D. Thesis, Florida State University.

Robinson, D. T. 2012, Function Data Analysis and Partial Shape Matching in the
Square Root Velocity Framework. Ph.D. Thesis, Florida State University.

Huang, W. 2014, Optimization Algorithms on Riemannian Manifolds with
Applications. Ph.D. Thesis, Florida State University.

Srivastava, A., Wu, W., Kurtek, S., Klassen, E. and Marron, J. S. (2011).
Registration of Functional Data Using Fisher-Rao Metric. arXiv:1103.3817v2
{[}math.ST{]}.

Tucker, J. D., Wu, W. and Srivastava, A. (2013). Generative models for
functional data using phase and amplitude separation. Computational Statistics
and Data Analysis 61, 50-66.

J. D. Tucker, W. Wu, and A. Srivastava, “Phase-Amplitude Separation of
Proteomics Data Using Extended Fisher-Rao Metric,” Electronic Journal of
Statistics, Vol 8, no. 2. pp 1724-1733, 2014.

J. D. Tucker, W. Wu, and A. Srivastava, “Analysis of signals under compositional
noise With applications to SONAR data,” IEEE Journal of Oceanic Engineering, Vol
29, no. 2. pp 318-330, Apr 2014.

Srivastava, A., Klassen, E., Joshi, S., Jermyn, I., (2011). Shape analysis of
elastic curves in euclidean spaces. Pattern Analysis and Machine Intelligence,
IEEE Transactions on 33 (7), 1415-1428.

S. Kurtek, A. Srivastava, and W. Wu. Signal estimation under random
time-warpings and nonlinear signal alignment. In Proceedings of Neural
Information Processing Systems (NIPS), 2011.

Wen Huang, Kyle A. Gallivan, Anuj Srivastava, Pierre-Antoine Absil. “Riemannian
Optimization for Elastic Shape Analysis”, Short version, The 21st International
Symposium on Mathematical Theory of Networks and Systems (MTNS 2014).

Cheng, W., Dryden, I. L., and Huang, X. (2016). Bayesian registration of functions
and curves. Bayesian Analysis, 11(2), 447-475.

W. Xie, S. Kurtek, K. Bharath, and Y. Sun, A geometric approach to visualization
of variability in functional data, Journal of American Statistical Association 112
(2017), pp. 979-993.

Lu, Y., R. Herbei, and S. Kurtek, 2017: Bayesian registration of functions with a Gaussian process prior. Journal of
Computational and Graphical Statistics, 26, no. 4, 894\textendash{}904.

Lee, S. and S. Jung, 2017: Combined analysis of amplitude and phase variations
in functional data. arXiv:1603.01775 {[}stat.ME{]}, 1-21.

J. D. Tucker, J. R. Lewis, and A. Srivastava, “Elastic Functional Principal
Component Regression,” Statistical Analysis and Data Mining, 10.1002/sam.11399, 2018.
\end{quote}


\chapter{Indices and tables}
\label{\detokenize{index:indices-and-tables}}\begin{itemize}
\item {} 
\DUrole{xref,std,std-ref}{genindex}

\item {} 
\DUrole{xref,std,std-ref}{modindex}

\item {} 
\DUrole{xref,std,std-ref}{search}

\end{itemize}


\renewcommand{\indexname}{Python Module Index}
\begin{sphinxtheindex}
\let\bigletter\sphinxstyleindexlettergroup
\bigletter{b}
\item\relax\sphinxstyleindexentry{boxplots}\sphinxstyleindexpageref{boxplots:\detokenize{module-boxplots}}
\indexspace
\bigletter{c}
\item\relax\sphinxstyleindexentry{curve\_functions}\sphinxstyleindexpageref{curve_functions:\detokenize{module-curve_functions}}
\indexspace
\bigletter{f}
\item\relax\sphinxstyleindexentry{fPCA}\sphinxstyleindexpageref{fPCA:\detokenize{module-fPCA}}
\item\relax\sphinxstyleindexentry{fPLS}\sphinxstyleindexpageref{fPLS:\detokenize{module-fPLS}}
\indexspace
\bigletter{g}
\item\relax\sphinxstyleindexentry{gauss\_model}\sphinxstyleindexpageref{gauss_model:\detokenize{module-gauss_model}}
\item\relax\sphinxstyleindexentry{geodesic}\sphinxstyleindexpageref{geodesic:\detokenize{module-geodesic}}
\indexspace
\bigletter{p}
\item\relax\sphinxstyleindexentry{pcr\_regression}\sphinxstyleindexpageref{pcr_regression:\detokenize{module-pcr_regression}}
\indexspace
\bigletter{r}
\item\relax\sphinxstyleindexentry{regression}\sphinxstyleindexpageref{regression:\detokenize{module-regression}}
\indexspace
\bigletter{t}
\item\relax\sphinxstyleindexentry{time\_warping}\sphinxstyleindexpageref{time_warping:\detokenize{module-time_warping}}
\item\relax\sphinxstyleindexentry{tolerance}\sphinxstyleindexpageref{tolerance:\detokenize{module-tolerance}}
\indexspace
\bigletter{u}
\item\relax\sphinxstyleindexentry{utility\_functions}\sphinxstyleindexpageref{utility_functions:\detokenize{module-utility_functions}}
\end{sphinxtheindex}

\renewcommand{\indexname}{Index}
\printindex
\end{document}